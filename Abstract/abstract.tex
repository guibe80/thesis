% ************************** Thesis Abstract *****************************
% Use `abstract' as an option in the document class to print only the titlepage and the abstract.
\begin{abstract}
\begin{singlespace}
One of the most prominent indicators of prosperous health is blood volume and flow. The basic  information within these health parameters may show cardiovascular problems or the advance  of further complications related to   other diseases like  diabetes. Peripheral vascular disease (PVD) and/or peripheral arterial disease (PAD) are sicknesses known to inadequate delivery of either arterial or venous blood towards the extremities. Such sickness  may trigger complications owing to the lack of transport of oxygen and nutrients, thus causing hypoxic events that may eventually prompt  to ischaemic tissue or even the loss of the compromised limb. Against this backdrop, a home monitoring blood or volume measurement system would be invaluable for prompt detection if blood flow is constantly  monitored. Bioelectrical impedance plethysmography (iPG) measures the changes in blood volume by injecting a small  amount of AC current into the body and after  measuring the potential created by blood flow. One of the benefits  of this technique is that it can measure a larger volume of tissue. This technique is easy to use as only needs four electrodes and has the potential to be miniaturised and being compact  for home utilization.

A bioelectrical impedance device including hardware and software was built  ready  to measure changes in blood volume/flow in the extremities with the capability of being utilised as a part of the home setting. The system was assessed  in an in-vivo environment with 9 participants by applying a mechanical constriction in the upper arm to assess the sensitivity of the instruments to changes in blood circulation . Three levels of blood restriction were applied: 1) below venous pressure 2) Amongst  venous and arterial pressure and 3) during total occlusion. Simultaneously, measurements from various  instruments like ECG, Doppler ultrasound, laser Doppler flowmetry and PPG were taken and compared to the measurements obtained from the iPG instrument and defining its correlation with the impedimetric signal. The bioimpedance device produced three signals including current injected to the participant, voltage from the forearm and a waveform containing the arterial pulses. It was discovered  that the instrument was sensible to recognize changes in all the three different occlusions applied to the participants. The signals acquired  from the basal impedance showed a distinct shift in rate when an arterial and a venous occlusion occurred, presenting information that it is possible to separate  between both kinds of flow disturbances. Furthermore, the plethysmographic signal captured by the instrument also distinguished  changes in the waveform morphology typical in large most of the participants during the similar period. These changes fluctuations provided additional further information that it might be possible to differentiate amongst  venous and arterial occlusions. By consolidating  the data obtained by the iPG device, it is possible to produce  an index that may   help to identify  problems in both circulatory systems in a substantial  volume of tissue.
\end{singlespace}
\end{abstract}
