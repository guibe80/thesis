%!TEX root = ../thesis.tex
%*******************************************************************************
%*********************************** First Chapter *****************************
%*******************************************************************************

\chapter{Introduction}  %Title of the First Chapter

\ifpdf
    \graphicspath{{Chapter1/Figs/Raster/}{Chapter1/Figs/PDF/}{Chapter1/Figs/}}
\else
    \graphicspath{{Chapter1/Figs/Vector/}{Chapter1/Figs/}}
\fi

Medicine at home is a concept that is looking into remotely monitor patients health with a minor intervention of medical practitioner. Thus, medical instruments capable of running safely at home requiring minimal mediation is the new trend in wearable devices. In the last years, a significant number of wearable technology or home care have been developed minimising unnecessary visits to medical centres and providing remote data to the practitioners.

One of the most valuable indicators of good health is the blood flow towards extremities. The information contains within this health parameter may indicate cardiovascular problems or the tendency of developing further issues related to other diseases like diabetes. An inadequate delivery of blood towards the extremities may lead to complications due to the lack of transport of oxygen and nutrients causing hypoxic events that may lead to ischemic tissue. A familiar name to denominate this kind of diseases is either peripheral vascular disease (PVD) or peripheral arterial disease (PAD). Nonetheless, the latter one focuses mostly when the blood reduction occurs in the arteries. Therefore, it is of great importance a method to also differentiate both kinds of events. The risk of developing PVD are hereditary factors and lifestyle choices. For instance, the combination of diabetes and smoking increments the chance of getting a PVD or complicating current diagnosis. If the illness is not treated on time, this may lead to the loss of the compromised limb.

These illnesses are progressive, thus if there was a method that could allow early detection or to prevent health deterioration, it could help to take corrective actions on time. Clinically, some methods or tests help clinicians to establish the development of these diseases. However, having a system that allows monitoring regularly changes in blood delivery would provide further information. Thus, better decisions could be made to create preventive methods to avoid new problems. 

Having a device capable of taking measurements at home collecting data would provide this additional health information. Some of the preferable characteristics of a device like that is that is simple to use, secure to use at home, portable and require minimum skill for operation. But one of the main considerations is that the instrument should be non-invasive or required intravenous aids to enhance measurements because then clinical personal might be needed.  

The blood delivery towards a limb can be estimated by estimating either the flow rate or the change of blood volume. Both parameters are well-recognised aid methods to study problems in the periphery. One of the most popular ways are instruments based on the Doppler technique. These can be applied with ultrasound or optical methods from where blood flow can be estimated. The Doppler methods have the potential to be used in a home environment, but it requires particular skills to place the sensor correctly. Moreover, it also measures a single point only focusing in either an artery, vein or the micro-circulation. 

Optical methods are quite accessible to measure changes of volume also known as Plethysmography and its relation with the absorption of oxygen in the deep tissue in a region of the body. Indeed, most of the new smartwatches available in the market measures heart rate based on this technology. Furthermore, blood flow rate can be estimated when using light in the near-infrared spectrum (NIRS) of the speed of movement of red cells with Laser Doppler techniques. The use of light is quite useful in a home setting, albeit its penetration is mostly superficial tissue. This is one of the limitations of this method, the volume of the tissue being monitored is quite small or shallow. It is not suitable for large volumes or deep tissue measurements. New methods like NIRS and optical imaging techniques look promising overcoming these drawbacks. Indeed, a system called Laser speckle contrast imaging (LSCI) is a suitable candidate to estimate blood circulation to a more comprehensive area of tissue with representation on a screen. However, again the signal obtained is just from the surface of the skin. 

The combination of two or more methods helps to improve the diagnostics of circulatory problems in the periphery. Moreover, it is possible to distinguish between arterial and venous problems by suppressing the effect of the venous return. This technique is known as venous occlusion plethysmography, which is applied to the limbs. By occluding on the upper section of the extremity, the venous blood is not able to return to the heart, raising the volume of the limb. By measuring, this increment is possible to gather information about the blood flow and the health of the veins in that part of the body. Therefore, this method needs two devices, one that blocks the venous blood commonly a pneumatic cuff and instrument to measure plethysmography. The latter can be achieved with different ways like chambers of air/water or strain gauges. Usually, using a chamber to measure plethysmography in a home setting can be quite cumbersome. Putting the arm in a tight cylinder filled with air or water and using a second hand to positioning other instruments do not fit within the principle of usability of a home device. On the other hand, strain gauges could be another option, but some sensors are filled with mercury, which is highly toxic. Some strain gauges can use another type of electrodes, but they are just confined to a small volume of tissue.

Bioelectrical impedance is a technique that can measure plethysmography and the amplitude of the arterial pulse. In short, the method called bioelectrical impedance plethysmography or simply impedance plethysmography (IPG) measures changes of blood volume by driving a tiny amount of current into the body and measuring the potential created by the blood flow. One of the advantages of this technique is that can measure a larger volume of tissue, not just a single vessel or superficial tissue. Due to the conductive nature of the human body, electric current flows smoothly through it and gaining a voltage measurement only a pair electrodes is required. This technology can be suitable for home, in fact, some manufacturers like Samsung \cite{simsense} are including this technology in their development of wearable products. As it can be seen, the miniaturisation of this technique is possible; it can be portable and easy to use. In fact, taking a useful measurement only needs four electrodes.

The information that bioelectrical impedance plethysmography can provide is quite rich and informative for circulatory problems. From this technology different methods have been developed to estimate plethysmography, blood flow and cardiac output using venous occlusion plethysmography or by just analysing the arterial pulses waveform. There have been developments where they concentrate on either the analysis of blood flow with venous occlusion or interpretation of the arterial pulses.

\section{Motivation and aims of the work}
The primary aim of this project is the design and implement and bioelectrical impedance device for the use in a home setting for the continuous monitoring of blood changes. It must work within the established limits of the standards. The instrument must comply with patient safety to use in a home setting. The device must be able to provide information about the changes in blood volume and blood flow when is possible. 

Another aim of the instrument is to include the capability of producing three different signals for further processing. It should provide an equivalent sensing value of current driven into the patient; voltage sensed and the arterial pulses. With this signals the device must proof that works for applications with venous occlusion plethysmography and plethysmography with free flow. For this, an experimental set-up has to be carried on evaluating the performance of the device under three different circulatory conditions: normal blood flow, venous occlusion and total occlusion. The data obtained will be correlated with the measurements of other portable instruments like ultrasound Doppler, photoplethysmography and laser Doppler flowmetry.

A secondary objective, the data obtained from the bioelectrical impedance plethysmography device must be compared to other instruments during the experiment. The information gathered will provide clues of how well impedance plethysmography performs during each circulatory of the artificial occlusion applied to the forearm. Besides, from the waveform acquired, the blood flow rate will be deducted researching how the changes in arterial and venous circulation affect the shape of the waveform.

A final objective is how the shape of the arterial pulses change at different points during normal flow, venous occlusion and total occlusion. This information will provide a clue of how to differentiate when a problem arises from the arterial or venous.

\section{Organisation of the thesis}
The thesis has been organised in different chapters with the clinical and technical background separated into two sections. This document covers the application of bioelectrical impedance technology in medicine. Therefore the chapter 2 will provide knowledge of the circulatory system including the blood, vessels and the cardiac cycle. The latter will describe how the changes in pressure and volume in the heart produce the well-known plethysmographic wave. Due to the experimental procedure carried out in an upper extremity, a detailed anatomy of the arm is provided for reference when naming some parts of this extremity. After understanding why monitoring vascular problems are useful for general health, this chapter will describe different instruments to measure changes in blood in a home setting. The studied methods will be focused on Doppler ultrasound, optical and venous occlusion plethysmography techniques. An introduction of how bioelectrical impedance plethysmography fits in the latter.

The chapter three will explain more in-depth how bioelectrical impedance works and is measured. First, an introduction of how much electrical current should be applied to the human body, followed by the electrical impedance principle and how to interpret complex impedance. After, the electrodes boundary is explained, detailing how ionic conduction takes place. The relation between electrical conduction and biology will be explored in the bioelectrical impedance section.  After that, the basic components of a bioelectrical impedance device are described, including different ways to measure it in the human body. Understood this part, it will be studied in more profundity the principle of bioelectrical impedance plethysmography and how the parallel model influences the measurement in cylinder shapes like in a human arm. Finally, the chapter will describe the kind of signals that can be obtained from a bioelectrical impedance plethysmography device on the forearm like basal impedance, venous occlusion plethysmography and the arterial pulse amplitude.

The chapter four will explain the work performed on the development of the bioelectrical impedance plethysmography device. The electrical characteristics of the equipment will be described including frequency of operation and maximum current delivery. A modular approach was utilised when designing the device so that each module will be described separately.  Each one will include designs and principle of operation. All the schematics will be found in the appendix A of this thesis. 

Chapter five describes the experimental procedure to gather data from participant's left arm. The instruments and locations of sensors are shown for the experiment. All the methods used will be presented in detail, including the different pressures applied to the cuff in the upper arm to alter the blood flow to the forearm. This chapter will also describe the nature of the data acquired, software front-end to capture data and post-processing. All the algorithms and mathematical formulas will be presented showing how blood flow can be estimated from some of the instruments and the bioelectrical impedance methods.  

\rvmynote{Check if I finally joined the chapter six and seven}
The following chapters describe the analysis of the signals obtained. The chapters six analyses the basal impedance from the measurement taken during the experiment. By isolating the basal readings during the experiments, errors produced by the instrument were analysed. Chapter seven describes the effect of occlusion on the basal impedance during the three different kinds of occlusion, below systolic pressure, in between systolic and diastolic and during total occlusion. Within this chapter, it will be calculated the blood flow using the theory and equations gathered in section four. Chapter eight will be concentrated on the analysis of the waveform shape of the arterial pulses during the experiment, including the changes in three different points in the plethysmographic waveform. From here, the calculation of blood flow from the waveform will be described. 

Chapters nine and ten present the data collected from additional instruments in the experimental procedure and the correlation with the bioelectrical impedance plethysmography (iPG). Here, the signals from iPG at different levels of occlusion will be correlated with the arterial flow of ultrasound Doppler, changes in amplitude pulses in PPG and the movement of red cells in laser Doppler flowmetry. 

Each episode will include a conclusion, depicting the most relevant information. However, the chapter eleven will show a general discussion of the performance of the bioelectrical impedance plethysmography device, and how correlates with the measurements from other instruments. Furthermore, the difference in arterial pulses during venous occlusion and partial arterial occlusion will provide a clue on how could be possible to differentiate between arterial and venous restriction of flow. 

\section{Work done}
A bioelectrical impedance device included hardware and software have been developed as part of the main work of the thesis. The PCB boards were designed in a modular system, were every board performs a specific task, for instance, wave generator, current injection and sensing modules. These are some of the components used in the design of the instrument. The device provides two different signals (current and voltage peak value) that can be used for further off-line processing. Using a USB DAQ card and a custom software application developed in LabView \cite{LabView:2016} were used to display and capture the waveforms coming from the instrument. Moreover, the interface also controlled some parameters of the apparatus such as frequency and trigger of the wave generator. An additional program was developed in Matlab \cite{MATLAB:2016} to post-processing the data captured. A custom graphic user interface was created to compare windowed data, apply filters, get statistical data and detect peaks and valleys.

From the data analysis point of view, different mathematical equations were applied as algorithms in Matlab to convert data into meaningful units like flow rate. Additional statistical analysis was used to compare the performance of the bioelectrical impedance device with other commercial instruments. The data from ultrasound Doppler was used to compared changes in arterial flow; the PPG signal was used to detect the changes in volume and laser Doppler flowmetry to movements in the microcirculatory bed.

\section{Novelty and contribution of the thesis}
Although some instruments are focused on measuring either basal or arterial pulses, the device proposed here is capable of detecting both components of the bioelectrical impedance.  Indeed, two channels were created to export each waveform. 

Although general bioelectrical impedance devices have been developed using quadrature demodulation, the instrument implemented here uses an envelope detector to demodulate the peak of the signal.

Although venous occlusion plethysmography is a well-known method, this can not be applied to all patients. Therefore, a way to detect problems in the venous circulation could be beneficial. From the waveform analysis of the arterial pulses, some changes during occlusions could provide a clue whether a problem in the venous flow is developing. 

