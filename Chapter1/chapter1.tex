%!TEX root = ../thesis.tex
%*******************************************************************************
%*********************************** First Chapter *****************************
%*******************************************************************************

\chapter{Introduction}  %Title of the First Chapter

\ifpdf
    \graphicspath{{Chapter1/Figs/Raster/}{Chapter1/Figs/PDF/}{Chapter1/Figs/}}
\else
    \graphicspath{{Chapter1/Figs/Vector/}{Chapter1/Figs/}}
\fi

Home diagnostics refers to technology that allows continuously monitoring patient's health with minimal or no need for intervention by a medical practitioner \cite{seal2007bringing} \rvmynote{Check this reference that says something about this.}. Diagnostic instruments capable of running safely at home requiring minimal mediation could potentially gather information for self assessment or to be sent remotely to medical experts for further analysis, and potentially for early detection of a condition. Quoting Seal et al. \cite{seal2007bringing} ''\textit{Early detection can limit the occurrence of acute events and complications that may lead to hospitalisation and extended hospital treatment}'', therefore it is desirable to have such monitor devices at home. (Indeed since its publication, a significant number of home healthcare and wearable devices have been developed minimising unnecessary visits to medical centres and providing remote data health practitioners.) \rvmynote{This can be eliminated because the word its publication}

One of the most valuable indicators of good health is blood volume and flow \cite{bloodcirculation}. The information contained within these health parameters may indicate cardiovascular problems or the tendency of developing further issues related to other diseases like diabetes. An inadequate delivery of blood towards the extremities may lead to complications due to the lack of transport of oxygen and nutrients causing hypoxic events that may lead to ischemic tissue. Some of the illnesses related to poor blood circulation in the extremities are peripheral vascular disease (PVD) and/or peripheral arterial disease (PAD) \cite{novo2004critical}. PVD can relate to hereditary factors or to lifestyle choices and, when combined with diabetes or smoking, the chances of getting the disease are dramatically increased, making frequent clinical visits for diagnostic tests imperative. If the illness is not treated on time, this may lead to the loss of the compromised limb. Thus, a home monitoring blood or volume measurement system would be invaluable for prompt detection. Regarding PAD, it focuses mostly on blood reduction in arteries and it is a good example of a case where a home diagnostic method that differentiates between either venous or arterial events would be of vital importance. 

These illnesses are progressive, thus if there was a method that could allow early detection or to prevent health deterioration, it could help to take corrective actions before any complication. Clinically, some techniques or tests help clinicians to establish the development of these diseases like ultrasound Doppler \cite{casey2008measuring} or venous occlusion plethysmography \cite{wilkinson2001venous}. However, using some of these instruments requires advanced specialised skills which make them difficult to operate at home. But having a system which regularly monitors changes in blood volume or flow would provide further information leading to better decisions for preventive treatments. 

(Having a device capable of collecting data measurements at home would provide this additional health information) \mynote{Remove this from}. According to an FDA initiative \cite{fdaini}, some of the preferable characteristics of a device like that are that is simple to use, secure to utilise at home and requires minimum skills for its operation. On top of this, portability would provide further benefits. But from the point of complexity, one of the leading considerations is that the instrument should be non-invasive. 

Blood delivery towards a limb can be estimated by measuring either the change of blood volume or flow rate. Both parameters are widely-used aid methods to study problems in the periphery \mynote{add reference}. One of the most popular ways are instruments based on the Doppler technique. These can be applied with ultrasound \cite{orekhova2013doppler} or optical \cite{fredriksson2007laser} methods from where blood flow can be estimated. The Doppler methods have the potential to be used in a home environment, but it requires particular skills to place the sensor correctly, often giving erroneous outcomes if the sensor is even slightly misplaced. Moreover, it measures a single point only focusing in either an artery, vein or micro-circulation. 

Devices using optical techniques are widely available to measure changes of volume also known as Plethysmography. Moreover, optical sensors can also provide information about the absorption of oxygen in the deep tissue in a region of the body \cite{holohan1996plethysmography}. Indeed, some of the new smartwatches available in the market measure heart rate based on this technology like the Apple iWatch \cite{culbert2017user}. Additionally, blood flow rate can be estimated when using light in the near-infrared spectrum (NIRS) \cite{van2001performance, harel2008near, de1993noninvasive, gurley2012noninvasive} or the speed of movement of red cells with Laser Doppler techniques \cite{dirnagl1989continuous,fredriksson2007laser}. The use of light is quite useful in a home setting, albeit its penetration can be quite superficial around muscle tissue. The lack of depth is one of the limitations of this method, the volume of the tissue being monitored is quite small or shallow \cite{bashkatov2005optical}. It is not suitable for large volumes or deep tissue measurements. New approaches like NIRS and optical imaging techniques look promising for overcoming these drawbacks. Indeed, a system called Laser speckle contrast imaging (LSCI) is a suitable candidate to estimate blood circulation to a more comprehensive area of tissue with representation on a screen \cite{briers2013laser}. However, again the signal obtained is just from the surface of the skin. Having a measurement from over a large would reduce the requirement of multiple and accurate point measurements, thus reducing the need of a skilled user for such a diagnostic device. 

(The combination of two or more methods helps to improve the diagnostics of circulatory problems in the periphery). Moreover, it is possible to distinguish between arterial and venous problems by suppressing the effect of the venous return \cite{wilkinson2001venous}. This technique is known as venous occlusion plethysmography, which is commonly applied to the limbs. By occluding on the upper section of the extremity, the venous blood is not able to return to the heart, raising the volume of the limb. By measuring, this increment is possible to gather information about the blood flow and the health of the veins in that part of the body \cite{wilkinson2001venous}. Thus, this method requires two devices, one that blocks the venous blood commonly a pneumatic cuff and instrument to measure plethysmography. The latter can be achieved with different ways like chambers of air/water or strain gauges. Nonetheless, using a chamber to measure plethysmography in a home setting can be quite cumbersome. Putting the arm in a tight cylinder filled with air or water and using a second hand to positioning other instruments do not fit within the principle of usability of a home device. On the other hand, strain gauges could be another option, but some sensors are filled with mercury, which is highly toxic. Some strain gauges can use another type of electrodes, but they are just confined to a small volume of tissue. \mynote{This could }

Bioelectrical impedance is a technique that can perform plethysmography and the amplitude of arterial pulses. In full, the method is called bioelectrical impedance plethysmography or simply impedance plethysmography (iPG) and, it measures changes of blood volume by driving a tiny amount of AC current into the body and measuring the potential created by blood flow \cite{bera2014bioelectrical}. One of the advantages of this technique is that it can measure a larger volume of tissue, not just a single vessel or superficial body part. A pair of electrodes is enough to drive current, and another couple measures the voltage drop in any part of the body. With this two elements (current and potential) is possible to calculate the impedance of the segment of the body contained within these electrodes. This technology can be suitable for home use, in fact, some manufacturers like Samsung \cite{simsense} are including this technology in their development of wearable products. This technique can be miniaturised, portable and easy to use. In fact, taking a useful measurement only needs four electrodes which can be part of a wearable garment thus removing the need for accurate placement which is a bottle neck of other monitoring methods.

The data coming from bioelectrical impedance plethysmography is quite rich and informative about circulatory problems. So, from this technology different methods have been developed to estimate plethysmography, blood flow and cardiac output. Some of these methods may require using venous occlusion plethysmography or analysing the arterial pulses waveform. No matter the technique used iPG has been proved to be a useful tool to investigate circulatory problems in the extremities \cite{bera2014bioelectrical, nyboer1974blood, mohapatra1979measurement, kyle2004bioelectrical, costeloe1980continuous, yamakoshi1980limb, porter1985measurement, corciova2011peripheral}

\section{Motivation of the thesis}
Bioelectrical impedance plethysmography has the potential to provide further information about blood circulation than other methods cannot supply. This technique has been proven to be useful in research and medical setting when measuring plethysmography and blood flow. However, designing a device that could be used at home which continuously monitors changes of blood is still uncommon. In principle, an instrument like this should be simple to use, safe for patients and preferably portable. In this dissertation, I am presenting a bioelectrical impedance plethysmography device which has the potential to be used at home under these conditions. The results obtained from the experimental work show not only a promising, protable, non-invasive , effective and potentially cheap instrument but also a way to differentiate changes in arterial and venous circulation. \mynote{Check this intro}  

\section{Aim and objectives of the work}
The aim of this project is the design and implementation of a bioelectrical impedance device for assessing iPG as a method to continuously monitors blood changes in a home setting. The apparatus must be safe and straightforward to use. For this, additional characteristics should include battery operation and low current delivery. From the quantification point of view, the device must be able to give data about the changes in blood volume and flow related to test conditions. 

A technical objective of the project is that the instrument should calculate the impedance reading from the direct computation of current and voltage readings from the patient. The signals produced by the device should be equivalent to the current delivered as well as the voltage from the forearm and the arterial pulses. With this information, the equipment must demonstrate that it works for applications like venous occlusion plethysmography and plethysmography with free flow. For this, an experimental set-up had to be carried on evaluating the performance of the device under three different circulatory conditions: normal blood flow, venous occlusion and total occlusion. 

After the collection of data, the objective was to correlate the signal of the bioelectrical impedance with other portable instruments like ultrasound Doppler, photoplethysmography and laser Doppler flowmetry. The information gathered will provide clues of how well impedance plethysmography performs during each kind of occlusion applied to the forearm. 

A final objective with the impedance plethysmography waveform obtained is to estimate changes of blood volume and flow rate per volume of tissue. \mynote{Convert this to a final 
}(Finally one of the objectives was) Additionally, performing a deeper signal analysis to identify shape changes of the arterial pulses at different points during normal flow, venous occlusion and total occlusion. 

\section{Novelty and contribution of the thesis}

\mynote{Introduce this terms basal and arterial pulses}
Although some instruments are focused on measuring either overall tissue impedance (including blood) \cite{mohapatra1979measurement, costeloe1980continuous, yamakoshi1980limb} or just arterial pulses \cite{corciova2011peripheral, porter1985measurement, brown1975impedance, marks1985computer}, the device proposed here is to my knowledge, the first device to demonstrate the detection of both elements of the bioelectrical impedance plethysmography. Indeed, two channels were created to export each waveform. 

Although general bioelectrical impedance devices have been developed using quadrature demodulation, the instrument implemented here uses an envelope detector to demodulate the peak of the signal.

Although venous occlusion plethysmography is a well-known method, this can not be applied to all patients. Therefore, a way to detect problems in the venous circulation could be beneficial. From the waveform analysis of the arterial pulses, some changes during occlusions could provide a clue whether a problem in the venous flow is developing.

\mynote{Table with the method}

To my knowledge at the time of writing this thesis,

\section{Work done}
A bioelectrical impedance device included hardware and software have been developed as part of the main work of the thesis. The PCB boards were designed in a modular system, were every board performs a specific task, for instance, wave generator, current injection and sensing modules. These are some of the components used in the design of the instrument. The device provides two different signals (current and voltage peak value) that can be used for further off-line processing. Using a USB DAQ card and a custom software application developed in LabView \cite{LabView:2016} were used to display and capture the waveforms coming from the instrument. Moreover, the interface also controlled some parameters of the apparatus such as frequency and trigger of the wave generator. An additional program was developed in Matlab \cite{MATLAB:2016} to post-processing the data captured. A custom graphic user interface was created to compare windowed data, apply filters, get statistical data and detect peaks and valleys.

From the data analysis point of view, different mathematical equations were applied as algorithms in Matlab to convert data into meaningful units like flow rate. Additional statistical analysis was used to compare the performance of the bioelectrical impedance device with other commercial instruments. The data from ultrasound Doppler was used to compared changes in arterial flow; the PPG signal was used to detect the changes in volume and laser Doppler flowmetry to movements in the microcirculatory bed.


\section{Organisation of the thesis}
The thesis has been organised in different chapters with the clinical and technical background separated into two sections. This document covers the application of bioelectrical impedance technology in medicine. Therefore the chapter 2 will provide knowledge of the circulatory system including the blood, vessels and the cardiac cycle. The latter will describe how the changes in pressure and volume in the heart produce the well-known plethysmographic wave. Due to the experimental procedure carried out in an upper extremity, a detailed anatomy of the arm is provided for reference when naming some parts of this extremity. After understanding why monitoring vascular problems are useful for general health, this chapter will describe different instruments to measure changes in blood in a home setting. The studied methods will be focused on Doppler ultrasound, optical and venous occlusion plethysmography techniques. An introduction of how bioelectrical impedance plethysmography fits in the latter.

The chapter three will explain more in-depth how bioelectrical impedance works and is measured. First, an introduction of how much electrical current should be applied to the human body, followed by the electrical impedance principle and how to interpret complex impedance. After, the electrodes boundary is explained, detailing how ionic conduction takes place. The relation between electrical conduction and biology will be explored in the bioelectrical impedance section.  After that, the basic components of a bioelectrical impedance device are described, including different ways to measure it in the human body. Understood this part, it will be studied in more profundity the principle of bioelectrical impedance plethysmography and how the parallel model influences the measurement in cylinder shapes like in a human arm. Finally, the chapter will describe the kind of signals that can be obtained from a bioelectrical impedance plethysmography device on the forearm like basal impedance, venous occlusion plethysmography and the arterial pulse amplitude.

The chapter four will explain the work performed in the development of the bioelectrical impedance plethysmography device. The electrical characteristics of the equipment will be described including frequency of operation and maximum current delivery. A modular approach was utilised when designing the device so that each module will be described separately.  Each one will include designs and principle of operation. All the schematics will be found in the appendix A of this thesis. 

Chapter five describes the experimental procedure to gather data from participant's left arm. The instruments and locations of sensors are shown for the experiment. All the methods used will be presented in detail, including the different pressures applied to the cuff in the upper arm to alter the blood flow to the forearm. This chapter will also describe the nature of the data acquired, software front-end to capture data and post-processing. All the algorithms and mathematical formulas will be presented showing how blood flow can be estimated from some of the instruments and the bioelectrical impedance methods.  

\rvmynote{Check if I finally joined the chapter six and seven}
The following chapters describe the analysis of the signals obtained. The chapters six analyses the basal impedance of all the participants. By isolating the basal readings during the experiments, errors produced by the instrument were analysed. Furthermore, it also describes the effect of occlusion on the basal impedance during the three different kinds of occlusion, below systolic pressure, in between systolic and diastolic and during total occlusion. The chapter eight will be concentrated on the analysis of the waveform shape of the arterial pulses during the experiment, including the changes in three different points in the plethysmographic waveform. From here, the calculation of blood flow from the waveform will be described. 

The chapters eight and nine present the data collected from additional instruments in the experimental procedure and the correlation with the bioelectrical impedance plethysmography (iPG). Here, the signals from iPG at different levels of occlusion will be correlated with the arterial flow of ultrasound Doppler, changes in amplitude pulses in PPG and the movement of red cells in laser Doppler flowmetry. 

Each episode will include a conclusion, depicting the most relevant information. However, the chapter ten will show a general discussion of the performance of the bioelectrical impedance plethysmography device, and how correlates with the measurements from other instruments. Furthermore, the difference in arterial pulses during venous occlusion and partial arterial occlusion will provide a clue on how could be possible to differentiate between arterial and venous restriction of flow. 


 

