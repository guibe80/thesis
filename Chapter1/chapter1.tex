%!TEX root = ../thesis.tex
%*******************************************************************************
%*********************************** First Chapter *****************************
%*******************************************************************************

\chapter{Introduction}  %Title of the First Chapter

\ifpdf
    \graphicspath{{Chapter1/Figs/Raster/}{Chapter1/Figs/PDF/}{Chapter1/Figs/}}
\else
    \graphicspath{{Chapter1/Figs/Vector/}{Chapter1/Figs/}}
\fi

Home diagnostics refers to technology that allows for continuous monitoring of the patient's health with minimal or no need for intervention on the part of a medical practitioner \cite{dittmar2004new}. Diagnostic instruments that are capable of running safely at home and require minimal mediation could potentially gather information for self assessment, or to be sent remotely to medical experts for further analysis, thereby potentially paving the way for an early detection of a medical condition.  Dittmar et al. \cite{dittmar2004new} have opined that home car or ambulatory monitoring may go a long way in helping citizens at risk because ''\textit{Early detection through long-term trend analysis will reduce the damage due to severe events dramatically}''; therefore, it is desirable to have such monitor devices at home.

One of the most valuable indicators of good health is blood volume and flow \cite{bloodcirculation}. The underlying information within these health parameters may indicate cardiovascular problems or the progress of further complications pertaining to other diseases such as diabetes. An inadequate delivery of blood towards the extremities may trigger complications owing to the lack of transport of oxygen and nutrients, thus causing hypoxic events that may eventually lead to ischemic tissue. Some of the illnesses pertaining to poor blood circulation in the extremities include peripheral vascular disease (PVD) and/or peripheral arterial disease (PAD) \cite{novo2004critical}. PVD can be attributed to hereditary factors or lifestyle choices; when combined with diabetes or smoking, the chances of contracting the disease increase drastically, making frequent clinical visits for diagnostic tests imperative. If left untreated, this illness may result in the loss of the compromised limb. Against this backdrop, a home monitoring blood or volume measurement system would be invaluable for prompt detection. PAD mainly focuses on blood reduction in arteries and signifies a good example of a scenario where a home diagnostic method that differentiates between venous or arterial events would be of vital importance. 

Given the progressive nature of these illnesses, if there was a method that could allow early detection or prevent the deterioration of health, it could accelerate the implementation of corrective actions before any complication. Clinically, some techniques or tests help clinicians establish the development of these diseases, such as ultrasound Doppler \cite{casey2008measuring} or venous occlusion plethysmography \cite{wilkinson2001venous}. However, some of these instruments require advanced specialised skills, which make them difficult to operate at home. Having a system that regularly monitors changes in blood volume or flow would provide further actionable information in order to facilitate better decisions for preventive treatments. 

According to an FDA initiative \cite{fdaini}, some of the preferable characteristics of a similar device include ease to use, secure enough to be utilised at home and requiring minimum skills for its operation. In addition, portability would impart additional benefits. However, from the standpoint of complexity, one of the primary considerations is that the instrument should be non-invasive. 

Blood delivery towards a limb can be estimated by measuring either the change of blood volume or flow rate. Both parameters are widely-used aid methods to examine critical problems in the periphery \cite{orchard1993assessment, hirsch2001peripheral}. In this regard, one of the most popular ways is instruments based on the Doppler technique. The instruments can be applied using ultrasound \cite{orekhova2013doppler} or optical \cite{fredriksson2007laser} methods using which, blood flow can be estimated. While the Doppler method is capable of being used in a home environment, it necessitates specific skills to place the sensor correctly, often leading to erroneous outcomes if the sensor is even slightly misplaced. Moreover, it measures a single point only, focusing on an artery, vein or micro-circulation. 

Devices involving optical techniques are widely available to measure changes of volume, also known as plethysmography. Moreover, optical sensors can uncover useful information about the absorption of oxygen in the deep tissue within an area of the body \cite{holohan1996plethysmography}. Indeed, some of the new smart watches available in the market, such as the Apple iWatch, measure heart rate based on this technology\cite{culbert2017user}. Additionally, blood flow rate can be estimated when using light in the near-infrared spectrum (NIRS) \cite{van2001performance, harel2008near, de1993noninvasive, gurley2012noninvasive} or the speed of movement of red cells with Laser Doppler techniques \cite{fredriksson2007laser, dirnagl1989continuous}. The use of light is quite useful in a home setting, although its penetration can be quite superficial around the muscle tissue \cite{bashkatov2005optical}. The lack of depth is one of the main limitations of this method; furthermore, the volume of the tissue being monitored is quite small or shallow \cite{bashkatov2005optical}. Moreover, it is not feasible to accommodate large volumes or deep tissue measurements. New approaches including NIRS and optical imaging techniques look promising in the context of overcoming these drawbacks. Indeed, a system referred to as Laser speckle contrast imaging (LSCI) is a suitable candidate to estimate blood circulation to a more comprehensive area of tissue with an adequate representation on a screen \cite{briers2013laser}. However, the signal obtained is merely from the surface of the skin. Having a measurement from over a large would reduce the need for multiple and accurate point measurements, thereby obviating the need of a skilled user for such a diagnostic device. 

Venous occlusion plethysmography (VOP) is another known method to assess circulatory problems in the periphery. This technique is primarily used to examine issues in the venous system by occluding on the upper section of the extremity. Performing this action blocks the return of venous blood into the heart by increasing the volume of the limb. By measuring and analysing this enlargement, it becomes possible to gather information about blood flow and the health of the veins in that specific part of the body \cite{wilkinson2001venous}. Thus, this method requires two devices one that blocks veins like a pneumatic cuff, and an instrument to measure plethysmography. The latter can be achieved by availing different technologies such as the chambers of air/water or strain gauges. Nonetheless, using a chamber to measure plethysmography in a home setting can be quite cumbersome. Positioning the arm in a tight cylinder filled with air or water and using the other hand to accommodate other instruments is incongruous with the principle of usability of a home device. Meanwhile strain gauges could be another viable option, but some sensors are filled with mercury, which is highly toxic. Some strain gauges can use another type of electrodes, but they are just confined to a small volume of tissue. Additionally, there are instances where a pressure cuff cannot be used in a patient. For instance, people with a ulnar nerve compression \cite{sy1981ulnar} and superficial phlebitis \cite{creevy1985phlebitis}. In such special cases, it is imperative to monitor changes of blood without obstructing the normal blood flow towards the extremity. 

Bioelectrical impedance denotes a technique that can perform plethysmography and also detects arterial pulses amplitudes. In entirety, this method is called bioelectrical impedance plethysmography, or simply impedance plethysmography (iPG). It measures the changes of blood volume by injecting a tiny amount of AC current into the body and subsequently measuring the potential created by blood flow \cite{bera2014bioelectrical}. One of the advantages of this technique is that it can measure a larger volume of tissue, as opposed to  a single vessel or superficial body part. A pair of electrodes is enough to drive current, with another couple measuring the voltage drop in any part of the body. Using the measurement of current and potential, it is possible to calculate the impedance of the segment of the body contained within these electrodes. This technology can be suitable for home use; in fact, some manufacturers like Samsung \cite{simsense} are already including this technology in their development of wearable products. This technique has the potential to be miniaturised and portable to ensure easy to use. As a matter of fact, taking a useful measurement only needs four electrodes that can be part of a wearable garment, thus obviating the need for accurate placement, which continues to be a bottleneck of other monitoring methods.

The data sourced from bioelectrical impedance plethysmography is quite rich and informative about circulatory problems. From this technology, different methods have been developed to estimate plethysmography, blood flow and cardiac output. Some of these methods may require the use of venous occlusion plethysmography or an analysis of the arterial pulses waveform. Regardless of the technique, iPG has proven to be a useful tool to investigate circulatory problems in the extremities \cite{bera2014bioelectrical, nyboer1974blood, mohapatra1979measurement, kyle2004bioelectrical, costeloe1980continuous, yamakoshi1980limb, porter1985measurement, corciova2011peripheral}

\section{Motivation of the thesis}
Bioelectrical impedance plethysmography has the potential to provide further information about blood circulation, something that is beyond the scope of  other methods. This technique has also been proven to be useful in research and medical settings, particularly when measuring plethysmography and blood flow. However, designing a device that could be used at home which continuously monitors the changes of blood remains an uncommon phenomenon. In principle, an instrument such as this should be simple to use, safe for patients and preferably portable. In this dissertation, I am presenting a bioelectrical impedance plethysmography device that has the potential to be used at home under these conditions. The results derived from the experimental work not only highlight a portable, non-invasive , effective and potentially cheap instrument, but also show the way forward to differentiate changes in arterial and venous circulation. 

\section{Aim and objectives of the work}
The aim of this project is the design and implementation of a bioelectrical impedance device for evaluating iPG as a method to continuously monitor blood changes within a home setting. The apparatus must be safe and straightforward to use. To that end, additional preferable characteristics should include battery operation and low current delivery. From the viewpoint of quantification, the device must be able to provide data about the changes in blood volume and flow related to test conditions. 

According to a technical objective of the project, the instrument should be able to calculate the impedance reading from the direct computation of current and voltage readings from the patient. The signals produced by this device should be equivalent to the current delivered as well as the voltage from the forearm and the arterial pulses. With this information, the equipment must demonstrate that it is feasible for applications including venous occlusion plethysmography and plethysmography with free flow. For that purpose, an experimental set-up had to be established in order to evaluate the performance of the device under three different circulatory conditions: normal blood flow, venous occlusion, and total occlusion. 

After data collection, the objective was to correlate the signal of the bioelectrical impedance with other portable instruments such as ultrasound Doppler, photoplethysmography and laser Doppler flowmetry. The accruing information will provide clues about the performance of impedance plethysmography during each kind of occlusion applied on the forearm. 

The final objective with the impedance plethysmography waveform obtained is to estimate the changes of blood volume and flow rate per volume of tissue. In addition, the aim is to perform a  deeper signal analysis so as to identify shape changes of the arterial pulses at different points during normal flow, venous occlusion and total occlusion. 

\section{Novelty and contribution of the thesis}

\mynote{Introduce this terms basal and arterial pulses}
Although some instruments are focused on measuring either overall tissue impedance (including blood) \cite{mohapatra1979measurement, costeloe1980continuous, yamakoshi1980limb} or only arterial pulses \cite{porter1985measurement, corciova2011peripheral, brown1975impedance, marks1985computer}, the proposed device, to the best of my knowledge, is the first one to demonstrate the detection of both elements of the bioelectrical impedance plethysmography. Indeed, two channels were created to export each waveform. 

Although general bioelectrical impedance devices have been previously developed using quadrature demodulation, the instrument implemented here utilises an envelope detector to demodulate the peak of the signal.

Although venous occlusion plethysmography is a well-known method, it cannot be applied on all patients. Therefore, a suitable way to detect problems in the venous circulation could be beneficial. From the waveform analysis of the arterial pulses, some changes during occlusions could provide a clue as to whether a problem is manifesting in the venous flow.

\section{Work done}
A bioelectrical impedance device, including hardware and software, has been developed as part of the main work of this thesis. The PCB boards were designed in a modular system, where every board performs a specific task - for instance, wave generator, current injection and sensing modules. These are some of the components used in the design of the instrument. The device provides two different signals (current and voltage peak value) that can be used for further off-line processing. A USB DAQ card and a custom software application developed in LabView \cite{LabView:2016} were used to display and capture the waveforms emitted by the instrument. Moreover, the interface controlled some parameters of the apparatus, such as frequency and trigger of the wave generator. Another program was developed in Matlab \cite{MATLAB:2016} to post-process the captured data. A custom graphic user interface was created to compare windowed data, apply filters, obtain statistical data and detect peaks and valleys.

From the viewpoint of data analysis, different mathematical equations were applied as algorithms in Matlab to convert data into meaningful units like flow rate. Additional statistical analysis was used in order to compare the performance of the bioelectrical impedance device with other commercial instruments. The data sourced from ultrasound Doppler was used to compare the changes in arterial flow; the PPG signal was used to detect the changes in volume and laser Doppler flowmetry to movements in the microcirculatory bed.


\section{Organisation of the thesis}
The thesis has been organised in different chapters, with the clinical and technical background being divided into two sections. This document covers the application of bioelectrical impedance technology in the field of medicine. Chapter 2 will provide knowledge about the circulatory system, including the blood, vessels and the cardiac cycle. The latter will describe how the changes in pressure and volume (in the heart) produce the well-known plethysmographic wave. Since the experimental procedure was carried out in an upper extremity, a detailed anatomy of the arm is provided for reference when naming some parts of this extremity. After gaining an understanding as to why monitoring vascular problems assume significance for general health, this chapter will elabore on the different instruments to measure changes in blood within a home setting. The studied methods will focus on Doppler ultrasound, optical and venous occlusion plethysmography techniques. An introduction of how bioelectrical impedance plethysmography fits in appropriately with the latter.

Meanwhile chapter three will describe, in an in-depth manner, about how bioelectrical impedance works and is measured. It begins with an introduction of how much electrical current should be applied to the human body, followed by the electrical impedance principle and  the manner in which complex impedance should be interpreted. Subsequently, the electrodes boundary is explained, detailing how ionic conduction takes place. The relationship between electrical conduction and biology will be explored in the bioelectrical impedance section.  Thereafter, the basic components of a bioelectrical impedance device are elaborated upon, including the different ways to measure it within a human body. This part will be studied in greater profundity to better understand the principle of bioelectrical impedance plethysmography and how the parallel model influences the measurement in cylinder shapes, as is the case with a human arm. Finally, the chapter will describe the kind of signals that can be obtained from a bioelectrical impedance plethysmography device on the forearm, including basal impedance, venous occlusion plethysmography and the arterial pulse amplitude.

Chapter four will explain the works performed towards the development of the bioelectrical impedance plethysmography device. The electrical characteristics of this equipment will be explained, including the frequency of operation and maximum current delivery. A modular approach was utilised when designing the device, so that each module will be elucidated separately.  Each one will be inclusive of designs and the principle of operation. All the schematics will be found in the appendix A of this thesis. 

Chapter five describes the experimental procedure to gather data from participant's left arm. The instruments and locations of sensors are illustrated for the experiment. All the methods used will be presented in detail, including the different levels of pressures applied to the cuff in the upper arm in order to alter the blood flow to the forearm. This chapter will also describe the nature of the data acquired, software front-end to capture data, and post-processing. All the algorithms and mathematical formulas will be presented to show how blood flow can be estimated from some of the instruments as well as the bioelectrical impedance methods.  

\rvmynote{Check if I finally joined the chapter six and seven}
The following chapters elaborate on the analysis of the signals obtained. Chapters six and seven analyse the basal impedance of all the participants. By isolating the basal readings during the experiments, the errors produced by the instrument were analysed. Furthermore, the effects of occlusion on the basal impedance during the three different kinds of occlusion - below systolic pressure, in between systolic and diastolic and during total occlusion - were also described. Chapter eight will focus on the analysis of the waveform shape of the arterial pulses during the experiment, including the changes occurring in three different points within the plethysmographic waveform. From this point, the calculation of blood flow from the waveform will be described. 

Meanwhile chapters eight and nine present the data collected from additional instruments in the experimental procedure in addition to the correlation with the bioelectrical impedance plethysmography (iPG). Here, the signals from iPG at different levels of occlusion will be correlated with the arterial flow of ultrasound Doppler, changes in amplitude pulses in PPG and the movement of red cells in laser Doppler flowmetry. 

Each episode will be inclusive of a conclusion, depicting the most relevant information. However, chapter ten will comprise of a general conclusion and achievements of the performance of the bioelectrical impedance plethysmography device, and the manner in which it correlates with the measurements from other instruments. Furthermore, the difference in arterial pulses during venous occlusion and partial arterial occlusion will provide a clue on  the possible way of differentiating between arterial and venous restriction of flow.



 

