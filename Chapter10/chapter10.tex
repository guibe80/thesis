%!TEX root = ../thesis.tex
%*******************************************************************************
%*********************************** Conclusions Chapter *****************************
%*******************************************************************************

\chapter{General conclusion and future work}  %Title of the First Chapter

\ifpdf
    \graphicspath{{Chapter10/Figs/Raster/}{Chapter10/Figs/PDF/}{Chapter10/Figs/}}
\else
    \graphicspath{{Chapter10/Figs/Vector/}{Chapter10/Figs/}}
\fi

The primary objective of this thesis was to present a home diagnostic system that would be capable of measuring changes in blood volume - and by extension flow - using bioelectrical impedance. Such a system was realised in \ref{chapter design} in the form of a modular device which carried out the required measurements and exhibited the capability of being used safely in a home setting. The device functioned as expected within the stated operational characteristics. Some of its key technological achievements and innovative features are highlighted as follows: 

\begin{itemize}
	\item Some of the studies described in the literature had used impedance devices that were not explicitly designed to measure blood-related alterations in limbs. For instance, some investigations used impedance cardiographers \cite{porter1985measurement, distefano1973bioelectrical, yamamoto1992impedance, couch1971noninvasive}, whereas other  circuits merely aimed to measure either basal impedance \cite{mohapatra1979measurement, yamakoshi1980limb, nyober1950electrical, yamakoshi1978admittance, yamamoto1992impedance} or arterial pulses \cite{mohapatra1979measurement, costeloe1980continuous, corciova2011peripheral, brown1975impedance, yamamoto1992impedance, wang2011development}. In comparison, the instrument presented in this thesis is primarily focused on measuring impedance within the range of the limbs' tissue \cite{gabriel1996dielectric}, simultaneously measuring the changes in the basal impedance and arterial pulses. 
	\item Although some studies used either their circuits or IC designs to measure  bioelectrical impedance, most of their techniques focused on quadrature demodulation and the average DC value of the signal \cite{yufera2002integrated, pallas1993bioelectric, min2000lock}. The method outlined in this thesis incorporates the envelope demodulation with a super diode circuit; this returns the peak value of the detected signals.
	\item Using envelope technique to measure the current amplitude and voltage from the limbs reduces the need for calibration. This instrument continuously measures the amplitude of the injected current and the voltage sensed from the limb's segment. Thus, the impedance was calculated directly from these two waveforms. 
	\item The iPG data presented in this literature are either basal impedance or arterial pulses. Some instruments extract the arterial pulses from  basal impedance, which results in low-quality waveforms \cite{mohapatra1979measurement, yamamoto1992impedance}. The instrument separates the arterial pulses from the forearm impedance via hardware. Thereafter, both signals are separately displayed  which provides the opportunity to analyse changes in both signals at the same time.
\end{itemize} 

The instrument's innovation presented in this thesis not only lies in the hardware design, but also in the analysis of  impedance plethysmography waveform. The protocol of this experimental procedure described in section \ref{chapter procedure} impacted the blood flow towards the limb. The instrument was able to effectively detect changes in the basal impedance as well as arterial pulses, which is evidenced in chapters \ref{chapter basal} and \ref{chapter apa} respectively. More importantly, clear changes could be detected in most participants when performing venous and partial arterial occlusions. According to the author, considering the changes observed during this experimental procedure, the following innovations can be derived:

\begin{itemize}
	\item Although studies performing occlusive techniques have been carried out, they have largely focused on the quantification of flow rate or volume heightened volume during VOP \cite{mohapatra1979measurement, costeloe1980continuous, yamakoshi1980limb}. However, none of these studies provided details on the rate of change during either venous or partial arterial occlusion. In contrast, the analysis presented in \ref{chapter basal} showed that the rate of change during partial arterial occurred occured more swiftly than venous occlusion. This slope change may be one of the first indicators as to when an arterial blockage may  occur when it is continuously monitored. 
	\item A study reported changes in the arterial pulses wave morphology during issues surfacing in the lymphatic system \cite{montgomery2011segmental}, but based on the author's understanding, no evidence of analysis was furnished during the mechanical occlusion. As evident in this thesis, it is clear that the morphology of this waveform varies in accordance to the type of occlusion applied to the upper arm. The data demonstrated that during both kinds of occlusions the systolic peak increased considerably. Thus, this might be an indicator of irregular blood flow towards the limb. Moreover, the data revealed that a difference in amplitude at the dicrotic notch as well as after pulse. During partial arterial occlusion, the amplitude of these points was seen to be considerably lower than venous occlusion. In fact, the information  demonstrated that during partial arterial occlusion, the post dicrotic amplitude tended to be of a lower magnitude when compared to dicrotic notch. Therefore, this is a clear differentiator between the two kinds of occlusion.
	\item To the best of the author's knowledge, no other study had previously been able to focus on the methods for early detection of venous and arterial circulatory problems using iPG. After a quantitative as well as qualitative analysis of the basal impedance and the arterial pulses, measuring the ratio between both signals in order to identify faster alterations in the circulation was considered. It was demonstrated that there was a shift in either basal impedance or APA morphology during each blockage. therefore, by combining these two sets of information, it is possible to detect early changes in the blood circulation.
\end{itemize} 

\section{Future work}
The present work showed an initial prototype to measure iPG within an upper limb along with the potential to detect circulatory problems in the limb. Nonetheless, some improvements in hardware and software would be able to refine the full prototype. Furthermore, some experiments can be suggested to confirm whether the body response triggers changes in the iPG waveform which allows for the differentiation between arterial and venous components of the impedance signal. 

\subsection{Improvements of the iPG device}
The device performed well during all the tests. However, there are items of interest that could be implemented in future studies. From the standpoint of design, there is room to miniaturise the instrument. A modular approach was used for this initial prototype. Nonetheless, designing an integrated PCB, which could include the MCU on board along with a  DAQ system, would simplify the design considerably. Some challenges may arise for the design of the board. During the PCB manufacturing, it was found that combining digital with analogue channels increased the crosstalk between signals. Therefore, special care must be undertaken when designing a board of these characteristics. 

In addition to this improvement, the device could accommodate a small set of batteries. The instrument included battery packs of \SI{12}{\volt} but it would be possible to achieve a better portability if lower power banks were installed. Furthermore, using a small current could be considered as an option to use a smaller battery bank. However, this may also lead to additional challenges. For instance, the impedance magnitude will get closer to the floor noise when using a lower electrical current. Therefore, the noise will cover the signal of interest. 

Another potential improvement to the design will be analysing the signal phase. As is the case with most bioelectrical impedance devices, the instrument only measured the magnitude of the impedance. However, additional information about volume or flow haemodynamics may be embedded within the phase shift data. Therefore, a channel dedicated to this phase could be an interesting add-on for future research. Nonetheless, there are quite a few challenges to achieve this. First, the phase shift at the frequency which the instrument was implemented at is only \SI{12}{\degree} \cite{jaffrin1979quantitative}. Therefore, the level of noise at that level can pose a serious challenge. Using a higher frequency could be an option, but then that would affect the actual part of the signal.

\subsection{Waveform analysis under different conditions}
The use of additional instruments provided insightful information on the changes in blood volume, arterial flow and microcirculation. However, it could be of high interest to compare the iPG waveforms and NIRS method. The latter can provide critical data about tissue oxygenation, and within the basal impedance, reading yields information about the amount of blood within the tissue. Therefore, the effectiveness of iPG in measuring perfusion changes can be established and compared to an optic method.

The analysis of waveforms has divulged differences when calculating the blood flow during both venous and partial arterial occlusion. The results that have been presented throughout the course of this experiment suggested an increase of  blood flow, which can only be explained by a vasodilation caused by a syncopal response. However, performing cold tests are recommended to investigate whether the narrowing of vessels also causes changes in the dicrotic notch point.

Analysing the waveform in  legs could also be of high interest. Some illnesses like diabetic food require constant monitoring of blood delivery towards the feet. The instrument could be adjusted to take the measurements within the impedance level so as to obtain a clear waveform signal. The changes in basal impedance and amplitude waveform could lead to an accurate indicator of poor perfusion in the compromised limbs.


