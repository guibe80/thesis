%!TEX root = ../thesis.tex
%*******************************************************************************
%*********************************** First Chapter *****************************
%*******************************************************************************

\chapter{Experimental Procedure}  %Title of the First Chapter
\label{chapter procedure}

\ifpdf
    \graphicspath{{Chapter4/Figs/Raster/}{Chapter4/Figs/PDF/}{Chapter4/Figs/}}
\else
    \graphicspath{{Chapter4/Figs/Vector/}{Chapter4/Figs/}}
\fi

This section will describe the protocol and signal obtained during the device performance test. City University London Research Ethics Committee approved the experimental procedure and protocol. Practice approved under reference \textit{''SREC 15-16 01 E 29 09 201''} of the \nth{11} of November 2015. 

In this study, nine healthy volunteers participated. In total, six males and three females aged \numrange{23}{37} years old (mean 28.77) participated in the study. As per regulation of the committee, only healthy participants took part of the research. Any participant with cardiovascular disease history did not take in the study. 

Previous to the recruitment of participants, they received documentation explaining the whole procedure. Once they agreed, the party returned the consent form signed to schedule the study. The experiments took place at the Research Centre for Biomedical Engineering of City, University of London. Upon arrival partaker acclimatised for \SI{10}{\min} which room temperature was \SI{22(2)}{\degreeCelsius}. During this period, it was clearly explained the experimental procedure to the attendant. Then the following steps took place.


%********************************** %First Section  **************************************
\section{Experimental procedure} %Section - 4.1
\label{section procedure 1}

After filling paperwork and completing acclimatisation different instruments were used to acquired physiological signals. These measurements included ECG, PPG, laser Doppler flowmeter, Doppler ultrasound and impedance plethysmography. Table \ref{table:instruments} describes the purpose of each instrument in this experiment. 

\begin{table}
	\caption{Instruments used during the study and function}
	\centering
	\label{table:instruments}
	\begin{tabu}{ccp{4.5cm}}
		\hline 
		\textbf{Instrument} & \textbf{Method} & \textbf{Measurement} \\\tabucline[2pt]{-}
		ECG & Sense of electrical charges in heart & Electrocardiogram \\\hline 
		PPG & Optical & Measurement of changes of volume in vascular bed \\\hline 
		LDF & Optical & Measurement flow in capillary bed (Cell level) \\\hline 
		iPG & Electrical & Measurement of changes of volume in a segment \\\hline
		Doppler Ultrasound & Electromagnetic & Measurement of flow speed \\\hline 
	\end{tabu}  
\end{table}

\mynote{check how to align this table properly}

%**********************************% Subsection 4.1.1  *************************************
\subsection{Instruments setup}
\label{section procedure 1.1}

First, iPG data were from the left arm of all assistants. However, this measurement requires recording forearm's segment volume by weighing length and circumference. In fact, it can be calculated using equation \ref{eq:v_e}. Additional, distance from the heart to shoulder, upper arm length, and shoulder to index finger length was also recorded. 

\begin{align}
	\label{eq:v_e}
	V_e =\frac{l \times (C_1^2+C_1 \times C_2 + C_2^2)}{(12 \times \pi)} \tagaddtext{[\si{\cubic\centi\meter}]}
\end{align}

where $V_e$ is segment's volume, $l$ is the length between sensing electrodes, $C_1$ is the circumference of elbow's electrode and $C_2$ is the circumference of wrist's electrode.

Second, all participants sat in a comfortable chair. Their left arm was resting on a soft cushion on a table next to a chair adjusted to collaborator's height. Then, blood pressure was taken using an automated instrument recording diastolic and systolic values. Then, participants placed four ECG leads themselves, forming an Einthoven triangle. According to the device's instructions, electrodes must be placed one on each shoulder and ankle. Then, leads were secured to the electrodes verifying that ECG signal was clean. The apparatus includes an output port that exports the waveform for further processing.

\mynote{Check process description for previous paragraph} 

\mynote{reference of the blood pressure device}

Following this step, a PPG probe was placed on the index finger. The PPG instrument used is a design of the research group known as Zen PPG. This equipment has two channels, but the experiment only required one channel. It provides two analogue outputs containing AC and DC components of the photoplethysmography waveform. 

\mynote{Check for the type of probe used. Nellcor?}
\mynote{I am not 100\% sure that has two outputs}

In the next step, the laser Doppler flowmetry probe was attached on the mid-section of the forearm. This device measures small changes in blood flow to the vascular bed under the skin. \nknote{the next sentences doesn't make sense counts with an analogue?}The instrument counts with an analogue port that extracts the waveform for further processing.  

\mynote{Check for maker of the Doppler Flowmetry device}

\mynote{Check a small paragraph about how this device works and what is useful for}

Later, the Doppler ultrasound probe was placed as close as possible to the radial artery. An audible signal is produced, which is comparable with the speed of blood flow at this point. For this, the probe head of the device requires to be placed at a fixed angle. Using a pole, the instrument's head was secured and placed on the party's wrist. The appliance also provides an external port for additional processing.  

\mynote{add a paragraph description of how Ultrasound Doppler works}

\mynote{adding the correct name of the pole}

Finally, impedance plethysmography electrodes were placed on the skin. For this, ECG electrodes were selected as they a provided good contact area which was required for the experiment to be carried out successfully. Current probes were placed following the path of the left radial artery, one below the elbow (brachial artery) and a second on top of the radial artery close to the wrist. \nknote{sensing wrong word}Sensing electrodes were located next to the previous electrodes towards the internal side of the forearm. The distance between sensing electrodes was recorded as the length of the volume segment. Additionally, the arm's circumference at these electrodes position was also taken. Using equation xx it is possible to estimate the volume of the \nknote{?segment which segment?}segment.  The device as described in Section xx provides two channels, impedance baseline and plethysmography waveform. 

\mynote{add reference to circumference equation}

\mynote{add reference to the previous chapter describing the device}


%********************************** %Second subection *************************************
\subsection{Data acquisition}
\label{section procedure 1.2}

As outlined in the previous section, all the instruments provided external ports for external data processing. These output ports were connected to a DAQ NI-6211 (National Instruments). This analogue to digital converter card provides 32* channels and a combined sampling rate of \SI{250}{\kilo\sample\per\second}. It is connected to a personal computer via USB port (Version 1.0). All the signals were sampled at \SI{1}{\kilo\hertz}.

\mynote{Maybe explain about the resolution of the DAQ. If this is a 16 bit the resolution is 2 to 16}

A virtual instrument using LabView was created for displaying, processing and storing raw data. Fig. Xx shows the front-end of the program. A custom virtual instrument was programmed showing four channels of waveforms. iPG, ECG, PPG and Ultrasound Doppler were those selected for portraying the signals.

\mynote{Add image of the front end of the LabVIEW instrument}

For display purposes, a band-pass filter was applied to the iPG and PPG signals. Nevertheless, the data recorded  was only from the unfiltered waveforms because post-processing was required. \nknote{expand on this and unsure if nevertheless is appropriate pls explain context to me to confirm} 

\mynote{check what filters were applied to the signal}

Two tabs provided settings and voltage adjustment for PPG signal. The participant tab required the following data: left arm dimensions, blood pressure and path name. At the end of the experimental procedure, all raw data was stored on an LVM file created by LabView. In the second tab, the VI adjusted the light intensity of the light PPG device. Usually a \SI{20}{\milli\ampere} is enough to provide a good quality signal. \nknote{what do you mean by this how is is good quality in what way?}

The main panel apart from just displaying waveforms also provide control buttons and a timer.  Pressing the start button starts a timer and waveforms begin to be stored in memory. Pushing on the same button again stops the timer and saves all the data in a local LVM file. 

\mynote{Add proper description of the DAQ used, including channels}
\mynote{Add reference of LabView in bibliography}

%********************************** %Third subection *************************************
\subsection{Experimental protocol}
\label{section procedure 1.3}

This experiment was aiming to look into waveform changes when an occlusion occurs. Achieving this is possible by applying a mechanical blockage using a standard blood pressure instrument. This device necessitates being pumped manually using an inflation bulb. The instrument also provides a gauge indicating the cuff's pressure level. The cuff was secured to the left biceps of the participant. 

The protocol required recording three different types of blood flow occlusion. The first kind of blood flow restriction was venous occlusion.  This obstruction can be produced by inflating the cuff usually between \SIrange{10}{20}{\mmHg} below diastolic pressure.\nknote{how do you know this works what is your reference} Thus, blocking the venous blood return from the forearm.  Overall, for each participant, the target pressure was \SI{20}{\mmHg} under their diastolic pressure recorded at the beginning of the session.\nknote{how can you confirm this why do you conclude this} 

The second class of blood flow restriction was partial arterial occlusion. This kind of blockage decreases the amount of arterial blood coming into the forearm but also stops venous blood return. As an illustration, this can be obtained by applying a mechanical compression to the upper arm. The desired value should be between diastolic and systolic pressure. Therefore, the objective was to occlude in the mean value between diastolic and systolic pressure. For that, it was calculated using the following equation.

\mynote{find a classy explanation about arterial blood occlusion, explaining that not allowing full compliance reduce flow}

\begin{align}
	\label{eq:meanpressure}
	P_m = \frac{P_d + P_s}{2} \tagaddtext{[\si{\mmHg}]}
\end{align}

where $P_d$ is diastolic pressure and $P_s$ is systolic pressure. 

The last kind of occlusion needed is total occlusion, which is possible to obtain by occluding above systolic pressure\nknote{reference}. In this experiment, blood flow was blocked by inflating the cuff above \SI{20}{\mmHg} of participant's systolic value recorded. This method of occlusion completely restricts the inflow and outflow of venous and arterial blood. Hence, it is expected that no change of volume takes place.

\subsubsection{Recording waveforms}

At this point, as soon as all the instruments were operating successfully, a small test took place. While waveforms were being shown on screen, \SI{2}{\min} of recordings were taken but not recorded yet. Participants were reminded to limit their movements during the experiment. 

As soon as all signals were readable and free of noise, the start button of the VI was pushed to initiate the study. In the beginning, \SI{5}{\min} of baseline recordings were obtained. These waveforms provided the standard control wave to compared with occluded waveforms. Then, as soon as the timer reached this time the cuff was inflated rapidly at \SI{20}{\mmHg} below diastolic pressure. Therefore, producing venous occlusion \nknote{swelling the forearm sounds wrong}and swelling the forearm. This level of blockage was held for \SI{3}{\min} followed by swiftly deflating the cuff. 

Similarly, it was required repeating the procedure with partial arterial occlusion.\nknote{?? previous sentence i dont know what you are trying to say} As found in the literature, after an occlusion happens \SI{5}{\min} are needed to restore blood pressure. For this reason, the cuff's air valve was left open bleeding all the air contained inside. After this period, the cuff was quickly inflated again until reaching the pressure calculate in equation \ref{eq:meanpressure}. The cuff's pressure was maintained at this level for \SI{3}{\min} proceeded by a quick pressure release. 

Last, for the purpose of total blood occlusion a similar method, was applied. One more time, \SI{5}{\min} of baseline waveforms were taken followed by a fleeting cuff inflation to \SI{20}{\mmHg} above diastolic pressure. This compression was kept for \SI{3}{\min}. At this point, maintaining this tourniquet effect can be painful after a couple of minutes. As a result, some of the helpers ended up re-accommodating and producing motion artefacts. 

Finally, the last \SI{5}{\min} of baseline waveforms were recorded to study the recovering effect. When time was up, the stop button was pushed on saving all the data onto an LVM file for further processing in Matlab. 

%********************************** % Swction section 4.2******************************************
\section{Data processing}
\label{section procedure 2}

Once data was stored in the local LVM file, it was post-processed in Matlab~\cite{MATLAB:2016}. To go into this in more detail, a GUI Matlab program was created capable of comparing signals, applying filters, as well as finding peaks and windowed sections of data. A front-end of the application can be seen in Fig. xxx.

\mynote{Create an image about the front-end of Matlab}

At first, importing an LVM file structure into Matlab was necessary. Due to the LVM file containing unwanted headers, a reading text command could not parse the data into Matlab easily. Moreover, sampling at \SI{1}{\kilo\hertz} created files of approximately \SI{300}{MB}. Importing this file size slowed down the workstation tremendously. For this reason, while importing into Matlab, data was decimated at a magnitude of ten. Consequently, each physiological waveform was reduced to less than \SI{1}{MB} of size. After all, having files at this size improved processing speed and used fewer resources. 

\mynote{create a table showing the structure of the LVM file}

This data decimation does not affect the waveform data. In fact, according to Nyquist rate, data was being oversampled. Physiological signals are between \SIrange{1}{2}{\hertz} for signals depending on heart rate. According to Nyquist up to \SI{4}{\hertz} would have provided enough sampling for the waveforms. Nevertheless, a sampling of \SI{100}{\hertz} is sufficient to provide a high resolution. Even so, having oversampled signals was not a random case. Indeed it allows implementing high order digital filters with sharp cut-off frequencies. 

\mynote{Add a little bit of Nyquist theorem for oversampling}

As soon as decimated data was imported into Matlab, the parsing program converted each waveform into a Matlab formatted files (.mat). Furthermore, this waveform data can be imported faster into the GUI than raw files. In essence, filters, windowing and data processing can be performed on-line saving in computing resources \nknote{dont understand previous sentence}. However, this data is not very significant until being converted into distinct values instead of volts.

%********************************** % Swction section 4.2.1******************************************
\subsection{Converting Data}
\label{section procedure 2.1}
When importing data into the GUI program, data needs to be converted into meaningful scales. All physiological data described in section \ref{section procedure 1.1} coming from the devices were in volts. Some data need conversions such as iPG, ultrasound and laser Doppler flowmetry.

\mynote{Add a section describing how the wave was extracted and converted into readable values. The algorithm used for this}

\subsubsection{Obtaining the impedance value}
Impedance plethysmography signals needed to be converted from volts into current and impedance respectively. The iPG device provides a channel for current sense $(V_{IDC})$, which corresponds to the peak voltage value of the current being delivered by the instrument. This value can be calculated by knowing the gain of the different stages of the circuits. According to the circuit schematic, current is being sensed by a \SI{10}{\ohm} resistor $(R_x)$.\nknote{dont unerstand previous sentence DNUPS} Then, this signal is amplified by an In-Ampnknote{make sure this part flows doesnt sound right} which gain $(G_1)$ was 276. Following this stage, the signal is halved by a half-wave rectifier. Later, this wave is fully rectified using a peak detection circuit. In the final analysis, the current value can be calculated using equation \ref{eq:current}

\begin{align}
	\label{eq:current}
	I = \frac{V_{IDC} \times 2}{R_x \times G_1} = \frac{V_{IDC} \times 2}{276 \times\ 10 \Omega} = \frac{V_{IDC}}{1380 \Omega} \tagaddtext{[\si{\ampere}]}
\end{align}

\mynote{add a reference toward the design of the device about the kind of data that ports provide. Current, AC and DC.}
\mynote{Add a section in the design of the device showing how current waveform is converted from AC signal to DC and then into a value.}
\mynote{add reference to the circuit that performs this task conversion task}

From the second channel $(V_{ZDC}$ of the iPG device the voltage coming from this port is equivalent to the impedance of the forearm. Firstly, it is wanted to know the real voltage value \nknote{doesnt make sense} $(V_z)$ from the forearm segment read by the impedance device. As can be seen from figure xxx, the signal was initially amplified by In-Amp (IAxx). The gain of this stage was set to 35.65. Then the signal was rectified by the super-diode circuit. These two are the only components affecting the signal path. In conclusion, the voltage can be calculated by using equation \ref{eq:vzdc}.

\mynote{add reference of a schematic showing the stages of the signal and a reference to the chip}

\begin{align}
	\label{eq:vzdc}
	V_z = \frac{V_{ZDC} \times 2}{35.65} \tagaddtext{[\si{\volt}]}
\end{align}

Finally, the impedance value can be computed by converting the voltage into impedance using Ohm's law. This can be achieved by replacing equation \ref{eq:current} and \ref{eq:vzdc} into \ref{eq:impedance}.

\begin{align}
	\label{eq:impedance}
	Z = \frac{v}{i}=\frac{V_z}{I} \tagaddtext{[\si{\ohm}]}
\end{align}

The third channel of the iPG device is the plethysmography waveform $V_{ZAC}$. This wave is an amplified version of the analogue signal contained within $V_{ZDC}$. Analysing the signal's path, it is noticeable that two different circuits process the waveform. First, the signal comes from $V_{ZDC}$, meaning that the gain of IA-xxx also affects the signal $V_{ZAC}$. Following the circuit, the signal is filtered by a band-pass filter described in section 3.1.6. This filter has a gain in DC of 142.42. Then, the voltage of the plethysmography wave can be found by combining both gains. The calculation can be performed with equation \ref{eq:ViPG}.

\begin{align}
	\label{eq:ViPG}
	V_{iPG} = \frac{V_{ZAC} \times 2}{142.42 \times 35.65}=\frac{V_{ZAC}}{2538.66} \tagaddtext{[\si{\volt}]}
\end{align}

\mynote{create a label and a link to the section 3.1.6}

Likewise, the impedance value of the dynamic signal can be calculated using Ohm's law. By replacing the voltage value of the plethysmography signal from equation \ref{eq:ViPG} and the value of the current from equation \ref{eq:current}.

\begin{align}
	\label{eq:zipg}
	Z_{iPG}=\frac{v}{i} = \frac{V_{iPG}}{I} \tagaddtext{[\si{\ohm}]}
\end{align}

\subsubsection{Converting ultrasound to flow}
\label{sectionDU}
The instrument used in this study was a Huntleigh model MD2 with sensor type VP8. The frequency of the transmitter is \SI{8}{\mega\hertz} which is ideal for blood flow estimation. The principle of this device relates to the frequency of a source to its velocity relative to a sensor~\cite{surgeonhand2002Hand}.  In other words, if an electromagnetic wave is transmitted at a fixed frequency and is reflected by a moving body, then the frequency of the received signal will be shifted~\cite{ht:MD2}.  

The device can detect moving blood cells within a vessel.  When a cell passes through the electromagnetic beam a frequency shift occurs, which is proportional to the blood flow velocity. Nevertheless, the Doppler shift is also affected by the angle of the head of the probe and the direction of the flow.\nknote{reference?}

Because average velocities found in the human body are within the audible frequency range, the device reproduces this soundnknote{reference}. An output channel also reproduces this audible signal for further processing. According to the specification of the instrument a \SI{3.5}{\volt} signal at the output is equivalent to a \SI{8}{\kilo\hertz} shift signal. Using this as a reference it is possible to calculate the shift frequency using equation \ref{eq:fshift}.

\begin{align}
	\label{eq:fshift}
	f_D = \frac{V_{DU} \times 8 KHz}{3.5V} \tagaddtext{[\si{\hertz}]}
\end{align}  

where $V_{DU}$ is the analogue output voltage of the instrument. 

According to the Doppler equation described in \ref{eq:doppler}, it is possible to find the velocity of a blood cell derived from its frequency shift. However, there is a correction angle $(\theta)$ between the ultrasound beam and the direction of blood flow. For this reason, the head of the sensor was positioned at a \SI{45}{\degree} angle to the radial artery on the wrist.

\begin{align}
	\label{eq:doppler}
	v = \frac{f_D \times C}{2 f_O \times Cos(\theta)} \tagaddtext{[\si{\meter\per\second}]}
\end{align}

where $f_D$ is the Doppler frequency, $C$ is the speed of sound assumed to be \SI{1540}{\meter\per\second}, $f_0$ is the oscillation frequency of the Doppler instrument in this case \SI{8}{\mega\hertz} and $\theta$ the angle between the ultrasound sensor and the target vessel which is \SI{45}{\degree}.\nknote{this paragraph doesnt make sense}

This angle position was locked using a laboratory stand and a clamp. The angle was fixed to \SI{45}{\degree} angle using a goniometer. Nevertheless, this is one of the shortcomings of this method. The angle can be estimated according to the position of the sensor's case but not to the surface of the crystal of the sensor. There is an incident error that cannot be quantified easily. However, for the event of this study, it was assumed that the angle is \SI{45}{\degree}.

Now the blood flow can be computed by converting velocity to flow if the cross section area of the vessel is known, as in equation \ref{eq:flow}. Not knowing this area precisely is another drawback of this method to calculate blood flow accurately. To know the real cross section area of a vessel requires imaging methods such as Doppler Ultrasonography. 

\begin{align}
	\label{eq:flow}
	\dot{Q} = v \times A \tagaddtext{[\si{\cubic\meter\per\second}]}
\end{align}

where $\dot{Q}$ is flow, $v$ is the velocity of the blood cell and $A$ the cross-sectional area of the radial artery.

However, blood flow can be estimated by using the average cross-sectional area of radial arteries in the population. A study has been found that males have slight larger radial arteries diameter than females (\SI{2.3(039)}{\mm} and \SI{2.11(029)}{\mm} respectively)~\cite{ashraf2010size}. These dimensions were used to calculate the blood flow in the present experiment. It is possible to convert this information into a more common scale of litre per minute by multiplying by 60 seconds and converting $m^3$ into litres. 

\begin{align}
\label{eq:flow_l/min}
\dot{Q} = v \times A \times 60 \times 1000 \tagaddtext{[\si{\cubic\meter\per\second}]}
\end{align}

\subsubsection{Converting LDF}
\label{section:ldf}
The Laser Doppler Flowmetry device utilised in the study was the xxx \mynote{Confirm the model of the LDF device}. LDF is a non-invasive optical method to estimate the blood perfusion in the microcirculation. This device uses the same Doppler principle as described in section \ref{sectionDU} but instead of sound, a beam of light is the source. Similarly, when a red blood cell scatters a light of beam, it produces a frequency shift~\cite{fredriksson2007laser}. 

LDF produces a blood perfusion signal that is comparable to the RBC perfusion or also known as flux. The units of this measurement are Blood Perfusion Unit (BPU) which is an arbitrary unit scale. BPU is the product of the mean number of moving blood cells in the small volume under the probe and the average velocity of moving blood cells. 

This instrument provides an output port that allows exporting the waveform. This connector provides a signal between \SIrange{0}{5}{\volt} which varies according to the flux signal. The configuration menu allows modifying the output scale. It provides three steps at a top limit of \SIlist{1000;500;100}{BPU} output for \SI{5}{\volt}. The measurements taken in this study showed being below \SI{100}{BPU}. Hence, using equation \ref{eq:BPU} converts the output voltage into BPU's.

\begin{align}
	\label{eq:BPU}
	BPU = \frac{V_{BPU} \times 100}{5 V}
\end{align}


%********************************** % Swction section 4.2.2******************************************
\subsection{Digital filtering}
\label{section procedure 2.2}

The signals obtained from the devices contained variable sources of noise. These included respiration, mains noise, motion, high-frequency noise from the iPG current source (\SI{30}{\kilo\hertz}). To remove these sources of different noise, filters were designed in Matlab for post-processing. The command \textit{designfilt} can create a variety of different filters. In the GUI, any user can apply any filter on demand to any signal available. Table \ref{table:filters} gives an overview of the filters that were designed and made available for processing the waveforms. 

\begin{table}[b]
	\caption{Filters available from the GUI}
	\centering
	\label{table:filters}
	\begin{tabular}{p{3.5cm} c c c c}
		\toprule
		\textbf{Filter}& \textbf{Order} & \textbf{Low cut frequency} & \textbf{High cut frequency} & \textbf{Other}\\
		\midrule
		Low-pass IIR & \nth{10} & -- & \SI{5}{\Hz} & --\\
		\midrule
		Band-pass IIR & \nth{10} & \SI{0.5}{\Hz} & \SI{5}{\Hz} & -- \\
		\midrule
		High-pass IIR & \nth{10} & \SI{0.5}{\Hz} & -- & --\\
		\midrule
		Band-stop IIR & \nth{10} & \SI{0.5}{\Hz} & \SI{5}{\Hz} & -- \\
		\midrule
		Savitzky-Golay & \nth{3} & -- & -- & Frame = 41\\
		\midrule
		Moving Average \newline (Simple) & -- & -- & -- & Lag = \SI{20}{\sec}\\
		\bottomrule
	\end{tabular}
\end{table}

According to the quality of the data, a specific kind of filter was applied to a distinct signalnknote{what do you mean}. For instance, raw iPG data contained high-frequency noise coming from the devices main carrier frequency. That sort of noise is easily removable using the high-pass filter available. Moreover, combining filters is possible. For instance, the iPG waveform is extractable from the impedance signal. In applying a mix of band-stop and band-pass filters is possible to recover the signal $iPG_{AC}$.

Similar processing is viable with the PPG signals. From the DC signal it is feasible to extract the AC waveform. Similarly, one can combine band-stop and band-pass filters.

In general, low-pass filters are useful to remove mains noise and high-frequency noises. Band-stop filters eliminate specific noise such as motion artefact and respiration but DC is still needed. A characteristic of this kind of noise is that it is low frequency and usually below heart rate frequency. \nknote{why is this relevant what does this mean?}High-pass filters are practical to remove DC signals but still keeping high-frequency characteristics.

\mynote{Add section describing how the signals were referenced to zero using the method}

%********************************** % Section 4.3 ******************************************
\section{Converting Impedance to Volume and Flow}
\label{section procedure 3}

After the conversion of impedance from volts, it is possible to compute changes of volume and flow. Chapter \ref{chapterdesign} describes in depth the governing equations. Additionally, the analysis of the impedance plethysmography waveform will provide further data of the blood's haemodynamics. Such as the relation between systole and diastole cycles with the iPG signal.

As explained in equation \ref{eq:dvdr} changes of impedances are equivalent to a fluctuation of volume.  The impedance calculated from equations \ref{eq:impedance} and \ref{eq:zipg} is the information needed to put into this Nyober's equation.\nknote{does this last sentence make sense?}

There are two methods to calculate blood flow. One of them is using venous occlusion plethysmography. Another one is the analysis of the dynamic changes of the impedance waveform. Both methods require using the resistivity derivative over time $dR/dt$. However, back projection of the Nyober's equation was used calculating flow in the impedance AC signal. The following section explains in detail the computation of these values.  

\mynote{Improve this paragraph, explaining both methods}

\subsection{Volume calculation during venous occlusion}
\label{section.4.3.1}
One of the methods to measure changes of volume involves occluding the venous return in one limb. This routine known as venous occlusion plethysmography is explained in detail in section xxx. This technique requires inflating a cuff below diastolic pressure to stop the blood's venous return in an extremity. Just like in the procedure described in section \ref{section procedure 1.3}. 

\mynote{double check a reference to Medical Background how this works. Create a reference to this section}

Due to the fact that with the occlusion venous blood is unable to return to the heart, blood starts to pool below cuff position in the limb. With regards to this experiment, a segment of the left arm was the subject of the study. The sensing electrodes are the boundary of the segment's geometry. The volume of this cylinder section is equivalent to the one calculated with equation \ref{eq:v_e}. 

Before occlusion, the base of the impedance does not change extremely over time. However, after the cuff is inflated, blood pools and conductivity of the limb segment increases. Therefore, impedance decreases within this volume. The change in the baseline of the impedance in time is equal to change in volume over time. A significant number of studies have been done in this field confirming the effectiveness of this method. 

\mynote{reference papers where venous occlusion plethysmography is used}
\mynote{Create a graph explaining how the gradient of the signal changes during venous occlusion}

The blood flow can be estimated from the basal impedance values, as seen in the case of the device designed port $Z_{DC}$. The change of volume can be calculated by selecting two points in time and applying equation \ref{eq:DVDT}.

\mynote{Confirm the nomenclature of the ports}

\begin{align}
	\label{eq:DVDT}
	\frac{\Delta V}{\Delta t}= \rho \frac{l^2}{R_B^2} \times \frac{\Delta R}{\Delta t}
\end{align} 

In order to make these values more meaningful, they need to be converted in a time interval per minutes. As well as, the limb blood flow should be expressed in units of \si{\milli\litre} per \SI{100}{\milli\litre} of limb section. In this case, the equation can be declared as \ref{eq:QL}.

\begin{align}
	\label{eq:QL}
	\dot{Q_L} = \bigg( \frac{1}{R_B} \bigg) \bigg( \frac{\Delta R}{\Delta t} \bigg) \times 60  \times 100  \tagaddtext{[\si{ml/min.100.ml}]}
\end{align} 

Where $\dot{Q_L}$ is the blood flow per \SI{100}{\milli\litre} of tissue, $R_B$ is the impedance base value in \si{\ohm}, $\Delta R_B$ is the change of impedance within ${\Delta t}$ in \si{\sec}.

%********************************** % Section 4.3.2 ******************************************
\subsection{Volume and flow calculation beat by beat}
\label{section procedure 3.2}
Impedance plethysmography also provides the opportunity to calculate the change in volume and the blood flow from every heart beat. Achieving this calculation is possible if there is a high-quality plethysmography waveform. For this reason, the impedance device includes an analogue port called $V_{ZAC}$. As described previously in section xxx, this port extracts the plethysmography waveform from the basal impedance $iPG_{AC}$. In fact, due to the filtering and amplification stages of the circuit, the signal has been amplified more than 2.500 times. Providing more characteristics and resolution of the waveform when digitalised.

\mynote{Verify the name of the port and also the section where should be referenced}

The iPG waveform as shown in figure xx is a classic example of a plethysmography wave. However, in impedance, this signal is inverted because an increase of blood represents a surge in conductivity. Hence, a reduction in impedance of the segment measured.  

\mynote{Add picture of waveform with reference points as expressed in the paragraphs RM1, RM2, RM3, RM4 and RM5}

The waveform obtained shows five particular reference points needed to calculate volume and blood flow. During the systolic process, two planes are evident in the plethysmography wave. The first point is the upslope of the signal $(R_{M1})$ which is present during the isovolumetric contraction of the heart.  This point marks the beginning of the plethysmography signal. Then, due to blood ejection from the heart, the aortic outflow increases the pressure in the circulatory system. Hence, a rise of blood flow is noticeable which is pronounced as the maximum amplitude in the plethysmography signal $(R_{M2})$. 

During the diastolic development of the heart cycle, the next reference point in the plethysmography wave which is observabl is the dicrotic notch. This stage occurs after a decrease in pressure. Previous research correlated this point with the echocardiography showed its relation with the isovolumetric relaxation of the heart.\nknote{prev sentence no sense} In the iPG wave collected by the device, there are two points within this region. \nknote{also doesnt make sense}First a dip $(R_{M3})$, followed by a peak of the post-dicrotic notch segment $R_{M4}$. 

\mynote{Find a reference between echocardiography, dicrotic notch and isovolumetric relaxation of the heart. Find paper relating heart cycle with the process during diastolic process} 

Finally, the cycle is completed with the last dip in the iPG waveform. Identified as $R_{M5}$, it also represents the beginning of the next volumetric cycle. 

Now that all the points of interest are identifiable in the waveform, it is possible to calculate the blood flow from the following set of equations. First, the average base impedance of the monitored segment during a pulse cycle can be obtained using equation \ref{eq:RB}.

\begin{align}
	\label{eq:RB}
	R_B = \frac{R_{M1}+R_{M5}}{2} \tagaddtext{[\si{\Omega}]}
\end{align}

Following this, one must extrapolate the reference points from the iPG pulse given by the Nyober back-projection~\cite{montgomery2011segmental} using equation \ref{eq:exht}. This value is equivalent to $dR/dt$ required \nknote{?required again wrong word?} by equation \ref{eq:dvdr}.

\begin{align}
	\label{eq:exht}
	EXHT = (R_{M3}-R_{M1}) + \frac{(t_{M3}-t_{M1})(R_{M2}-R_{M3})}{t_{M3}-t_{M2}} \tagaddtext{[\si{\Omega}]}
\end{align}

The heart rate is important in providing meaningful units to the measurements. Using the time difference during an iPG pulse, it is possible to obtain the HR as shown by equation \ref{eq:hr}, in which units are beats per minute.

\begin{align}
	\label{eq:hr}
	HR = \frac{60}{(t_{M5}-t_{M1})}  \tagaddtext{[\si{bpm}]}
\end{align}

Next replacing \nknote{?} $EXHT$ and $HR$ from equations \ref{eq:RB} and \ref{eq:exht} respectively, into Nyober's equation (see \ref{eq:Nyober}) is possible to calculate the blood flow.\nknote{sentence doesnt make sense} However, computing it as blood flow withnknote{? with?} units (\si{\milli\litre\per\minute}), the HR must be included in the equation. In the end, Nyober's equation can be re-written as \ref{eq:bf}. 

\begin{align}
	\label{eq:bf}
	BF = HR \times EXHT \times \rho \frac{l^2}{R_B^2} \tagaddtext{[\si{\ml\per\min}]}
\end{align}

where $BF$ is the blood flow expressed in \si{\milli\litre\per\min}, $\rho$ is the specific resistivity of blood \SI{150}{\ohm\per\cm}~\cite{mohapatra1981non, nyober1950electrical}, and $l$ is the distance between sensing electrodes.

Blood flow could also be noted as the blood flow per total arm segment volume found in equation \ref{eq:v_e}.  In other words, it can be expressed as the blood flow passing through the sensing electrodes per litre of volume. In order to achieve this, one can use equation \ref{eq:bfve}.

\begin{align}
	\label{eq:bfve}
	BFVE = \frac{BF}{V_e} \times \text{1000 ml} \tagaddtext{[\si{\ml\per\min.\litre}]}
\end{align}

where BFVE is the blood flow per litre of volume \si{\ml\per\min.\litre} , BF is the blood flow found in \ref{eq:bf}, $V_e$ is the forearm's segment volume in \si{\cubic\cm} from equation \ref{eq:v_e} and \SI{1000}{ml} is the scaling factor to convert the volume into litre.

In conclusion, it is possible to compute the blood flow of \SI{100}{\milli\litre} of tissue. This can be obtained as the blood flow of the whole segment by \SI{100}{\milli\litre} of volume.\nknote{????2 sentences} Hence, the equation \ref{eq:bfpct} provides the blood flow in the units of interest. 

\begin{align}
	\label{eq:bfpct}
	BF_{100ml} = \frac{BF}{V_e} \times \text{100 ml} \tagaddtext{[\si{\bfv}]}
\end{align}

%********************************** % Section 4.3.3 ******************************************
\subsection{Additional haemodynamic parameters analysis beat by beat}
\label{section4.3.3}
From the landmarks \nknote{?landmarks} detected in the data, additional haemodynamic parameters are deductible. Three reference points within each pulse can provide rheographic information. The first one is the rheographic index of the pulse volume, which is obtained from the peak value of the systolic section of the plethysmography waveform $R_{M2}$ in impedimetric form. It provides information about the height of the peak compared to the baseline of the waveform. It can be calculated using equation \ref{eq:A}.

\begin{align}
	\label{eq:A}
	A = R_{M2} - \frac{(t_{M2}-t_{M1}) \times ((R_{M5}-R_{M1})}{((t_{M5}-t_{M1})} \tagaddtext{[\si{\ohm}]}
\end{align}

Similar data can be obtained by analysing the diacrotic notch reference points. Using these points the height of the plethysmography waveform at these points is also calculated. These index points will prove valuable when analysing the waveform during venous and partial arterial occlusions. The points B and C can be calculated using equations \ref{eq:B} and \ref{eq:C} respectively.  

\begin{align}
	\label{eq:B}
	B = R_{M3} - \frac{(t_{M3}-t_{M1}) \times ((R_{M5}-R_{M1})}{((t_{M5}-t_{M1})} \tagaddtext{[\si{\ohm}]}
\end{align}

\begin{align}
	\label{eq:C}
	C = R_{M4} - \frac{(t_{M4}-t_{M1}) \times ((R_{M5}-R_{M1})}{((t_{M5}-t_{M1})} \tagaddtext{[\si{\ohm}]}
\end{align}

By combining the previous indexes, it is viable to calculate ratios against the peak at index $A$. When compared, \nknote{/when compared makes sense?} $C/A$ will provide information about the arteriolar tone. Likewise, $B/A$ contains a clue about the venular tone. These values represent the percentage in ohms between those reference points. In the results section, these values are labelled $DCI$ for arteriolar tone and $DSI$ for venular tone.

\mynote{Find information about arteriolar and venular tone. Add information about it in the literature review}

Also, the area under the curve during the arterial and venous cycle can be computed. Calculating the integral value from $t_{M1}$ to $t_{M3}$ provides the full contribution of the arterial process during the cycle. The area under the curve between $t_{M3}$ to $t_{M5}$ produces the input of the venous reporting period. From the mathematical point of view, these can be calculated using equations \ref{eq:ST}.

\begin{align}
	\label{eq:ST}
	ST_1 = \int_{t_{M1}}^{t_{M2}} f(Z) dZ \quad and \quad ST_2 = \int_{t_{M3}}^{t_{M5}} f(Z) dZ 
\end{align}

This computational work was performed in Matlab using the command \textit{''trapz''}. This code performs the trapezoidal numerical integration within the time intervals described previously. Returning the area under the curve in ohms.



%********************************** %Nomenclature found  *************************************
\nomenclature[z-ecg]{ECG}{Electrocardiography}
\nomenclature[z-ppg]{PPG}{Photoplethysmography}
\nomenclature[z-ipg]{iPG}{Impedance Plethysmography}
\nomenclature[z-daq]{DAQ}{Data Acquisition Card}
\nomenclature[z-usb]{USB}{Universal Serial Bus}
\nomenclature[z-vi]{VI}{Virtual Instrument}
\nomenclature[z-gui]{GUI}{Graphic User Interface}
\nomenclature[z-iir]{IIR}{Infinite Impulse Response}
\nomenclature[z-c]{C}{Speed of sound}
\nomenclature[z-ldf]{LDF}{Laser Doppler Flowmetry}
\nomenclature[z-bpu]{BPU}{Blood Perfusion Unit}
\nomenclature[z-du]{DU}{Doppler ultrasound}
\nomenclature[g-p]{$\pi$}{ $\simeq 3.14\ldots$}  
\nomenclature[g-r]{$\rho$}{Blood resistivity $\simeq$ \SI{150}{\ohm\per\centi\meter}}  
\nomenclature[g-t]{$\theta$}{Angle of incidence}
