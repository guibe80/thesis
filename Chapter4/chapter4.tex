%!TEX root = ../thesis.tex
%*******************************************************************************
%*********************************** First Chapter *****************************
%*******************************************************************************

\chapter{Experimental Procedure}  %Title of the First Chapter

\ifpdf
    \graphicspath{{Chapter4/Figs/Raster/}{Chapter4/Figs/PDF/}{Chapter4/Figs/}}
\else
    \graphicspath{{Chapter4/Figs/Vector/}{Chapter4/Figs/}}
\fi


This section will describe the protocol and signal obtained during the device performance test. City University London Research Ethics Committee approved the experimental procedure and protocol. Practice approved under reference \textit{''SREC 15-16 01 E 29 09 201''} of the \nth{11} of November 2015. 

In this study, nine healthy volunteers participated. In total, six males and three females aged \numrange{23}{37} years old (mean 28.77) participated in the study. As per regulation of the committee, only healthy participants took part of the research. Any participant with cardiovascular disease history did not take in the study. 

Previous to the recruitment of participants, they received documentation explaining the whole procedure. Once they agreed, the party returned the consent form signed to schedule the study. The experiments took place at the Research Centre for Biomedical Engineering of City, University of London. Upon arrival partaker acclimatised for \SI{10}{\min} which room temperature was \SI{22}{\degreeCelsius}\SI{\pm 2}{\degreeCelsius}. During this period, it was clearly explained the experimental procedure to the attendant. Then the following steps took place.


%********************************** %First Section  **************************************
\section{Experimental procedure} %Section - 4.1
\label{section4.1}

After filling paperwork and completing acclimatisation different instruments were used to acquired physiological signals. These measurements included ECG, PPG, laser Doppler, ultrasound Doppler and impedance plethysmography. 


%********************************** % Third Section  *************************************
\subsection{Instruments setup}
\label{section4.1.1}

First, iPG data were from the left arm of all assistants. However, this measurement requires recording forearm's segment volume by weighing length and circumference. In fact, it can be calculated using equation \ref{eq:Volume}. Additional, distance from the heart to shoulder, upper arm length, and shoulder to index finger length was also recorded. 

\begin{align}
\label{eq:Volume}
V_e =\frac{l(C_1^2+C_1 C_2 + C_2^2)}{(12\pi)}
\end{align}

where $V_e$ is segment's volume, $l$ is the length between sensing electrodes, $C_1$ is the circumference of elbow's electrode and $C_2$ is the circumference of wrist's electrode.

Second, all participants sat in a comfortable chair. Their left arm was resting on a soft cushion on a table next to a chair adjusted to collaborator's height. Then, blood pressure was taken using an automated instrument recording diastolic and systolic values. Then, participants placed four ECG leads themselves, forming an Einthoven triangle. According to the device's instructions, electrodes must be placed one on each shoulder and ankle. Then, leads were secured to the electrodes verifying that ECG signal was clean. The apparatus includes an output port that exports the waveform for further processing.

\todo{Check process description for previous paragraph} 

\todo{reference of the blood pressure device}

Following this step, a PPG probe was placed on the index finger. A PPG instrument designed by the research group known as Zen PPG was used for this purpose. This equipment has two channels, but only one was required for the experiment procedure. It provides two analogue outputs containing AC and DC components of the photoplethysmography waveform. 

\todo{Check for the type of probe used. Nellcor?}

In the next step, the laser Doppler flowmetry probe was attached on the forearms' mid-section. This device measures small changes in blood flow over the vascular bed under the skin. The instrument counts with an analogue port that extracts the waveform for further processing.  

\todo{Check for maker of the Doppler Flowmetry device}

\todo{Check a small paragraph about how this device works and what is useful for}

Later, the ultrasound Doppler probe was placed close as possible to the radial artery. It produces an audible signal comparable with blood's speed at this point. For this, the device 's probe head requires being placed at a fixed angle. Using a pole the instrument's head was secured and placed on the party's wrist. The appliance also provides an external port for additional processing.  

\todo{add a paragraph description of how Ultrasound Doppler works}

\todo{adding the correct name of the pole}

Finally, impedance plethysmography electrodes were allocated. For this, ECG electrodes provided good contact area required for the experiment. Current probes were placed following the path of the left radial artery, one below the elbow (brachial artery) and a second on top of the radial artery close to the wrist. Sensing electrodes were located next to the previous electrodes towards the internal side of the forearm. The distance between sensing electrodes was recorded as the length of the volume segment. Also, arm's circumference at these electrodes position also was taken. Using equation xx is possible to estimate the volume of the segment.  The device as described in Section xx provides two channels, impedance baseline and plethysmography waveform. 

\todo{add reference to circumference equation}

\todo{add reference to the previous chapter describing the device}

\subsection{Data acquisition}

\label{section4.1.2}

As outlined in the previous section, all the instruments provided external ports for external data processing. These output ports were connected to a DAQ NI-6211 (National Instruments). This analogue to digital converter card provides 32* channels and a combined sampling rate of \SI{250}{\kilo\sample\per\second}. It is connected to a personal computer via USB port (Version 1.0). All the signal were sampled at \SI{1}{\kilo\hertz}.

A virtual instrument using LabView was created for displaying, processing and storing raw data. Fig. Xx shows the front-end of the program. A custom virtual instrument was programmed showing four channels of waveforms. iPG, ECG, PPG and Ultrasound Doppler were used for displaying signals.

\todo{Add image of the front end of the LabVIEW instrument}

For display purposes, a band-pass filter was applied to the iPG and PPG signals. Nevertheless, the data recorded only was from the unfiltered waveforms because post-processing was required.  

\todo{check what filters were applied to the signal}

Two tabs provided settings and voltage adjustment for PPG signal. The participant tab required the following data: left arm dimensions, blood pressure and path name. At the end of the experimental procedure, all raw data was stored on an LVM file created by LabView. In the second tab, the VI adjusts the light intensity of the light PPG device. Usually a \SI{20}{\milli\ampere} is enough to provide a good quality signal. 

The main panel apart from just displaying waveforms also provides control buttons and a timer.  Pressing the start button starts a timer and waveforms begin to be stored in memory. When the same button is pushed on again, the program saves all the data in a local LVM file. 

\todo{Add proper description of the DAQ used, including channels}
\todo{Add reference of LabView}

%********************************** %Second Section *************************************

\subsection{Experimental protocol}

\label{section4.1.3}

This experiment was aiming to look into waveform changes when an occlusion occurs. Achieving this is possible by applying a mechanical blockage using a standard blood pressure instrument. This device requires being pumped manually using an inflation bulb. The instrument also provides a gauge indicating cuff's pressure level. The cuff was secured to the left biceps of the participant. 

The protocol required recording three different types of blood flow occlusion. The first kind of blood flow restriction was venous occlusion.  This obstruction can be produced by inflating the cuff usually between \SIrange{10}{20}{\mmHg} below diastolic pressure. Thus, blocking the venous blood return from the forearm.  Overall, for each participator the target pressure was \SI{20}{\mmHg} under their diastolic pressure recorded at the beginning of the session. 

The second class of blood flow restriction was partial arterial occlusion. This kind of blockage decreases the amount of arterial blood coming into the forearm but also stops venous blood return. As an illustration, this can be obtained by applying a mechanical compression to the upper arm. The desired value should be between diastolic and systolic pressure. Therefore, the objective was to occlude in the mean value between diastolic and systolic pressure. For that, it was calculated using the following equation.

\todo{find a classy explanation about arterial blood occlusion, explaining that not allowing full compliance reduce flow}

\begin{align}
\label{eq:meanpressure}
P_m = \frac{P_d + P_s}{2}
\end{align}

where $P_d$ is diastolic pressure and $P_s$ is systolic pressure. 

The last kind of occlusion needed is total occlusion, which can be obtained by occluding above systolic pressure. In this experiment, blood flow was blocked inflating the cuff above \SI{20}{\mmHg} of participant's systolic value recorded. This kind occlusion completely restricts the inflow and outflow of venous and arterial blood. Hence, it is expected that not change of volume takes place.

\subsubsection{Recording waveforms}

At this point, when it was confirmed that all the instruments were operating a small test took place. While waveforms were being shown on screen, \SI{2}{min} of recordings were taken but not recorded yet. Participants were reminded to limit their movements during the experiment. 

As soon as all signals were readable and free of noise, the start button of the VI was pushed to initiate the study. In the beginning, \SI{5}{\min} of baseline recordings were obtained. These waveforms provided the standard control wave to compare with occluded waveforms. Then, as soon as the timer reached this time the cuff was inflated rapidly at \SI{20}{\mmHg} below diastolic pressure. Therefore, producing venous occlusion and swelling the forearm. This level of blockage was held for \SI{3}{\min} followed by a swift cuff's deflation. 

Similarly, it was required repeating the procedure with partial arterial occlusion. As found in the literature, after an occlusion happens \SI{5}{\min} are needed to restore blood pressure. For this reason, cuff's air valve was left open bleeding all the air contained inside. After this period, the cuff was quickly inflated again until reaching the pressure calculate in equation \ref{eq:meanpressure}. The cuff's pressure was maintained at this level for \SI{3}{\min} proceeded by a quick pressure release. 

Last, for the purpose of total blood occlusion a similar method, was applied as before. One more time, \SI{5}{\min} of baseline waveforms were taken followed by a fleet cuff inflation to \SI{20}{mmHg} above diastolic pressure. This compression was maintained for \SI{3}{\min}. At this point, maintaining this crushing effect can be painful after a couple of minutes. As a result, some of the helpers ended up re-accommodating and producing motion artefacts. 

Finally, the last \SI{5}{\min} of baseline waveforms were recorded to study the recovering effect. When time was up the stop button was pushed on saving all the data into an LVM file for further processing in Matlab. 


%********************************** %Nomenclature found  *************************************
\nomenclature[z-ecg]{ECG}{Electrocardiography}
\nomenclature[z-ppg]{PPG}{Photoplethysmography}
\nomenclature[z-rc]{iPG}{Impedance Plethysmography}
\nomenclature[z-daq]{DAQ}{Data Acquisition Card}
\nomenclature[z-usb]{USB}{Universal Serial Bus}
\nomenclature[z-vi]{VI}{Virtual Instrument}
