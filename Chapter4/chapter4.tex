%!TEX root = ../thesis.tex
%*******************************************************************************
%*********************************** First Chapter *****************************
%*******************************************************************************

\chapter{Experimental Procedure}  %Title of the First Chapter

\ifpdf
    \graphicspath{{Chapter4/Figs/Raster/}{Chapter4/Figs/PDF/}{Chapter4/Figs/}}
\else
    \graphicspath{{Chapter4/Figs/Vector/}{Chapter4/Figs/}}
\fi


This section will describe the protocol and signal obtained during the device performance test. City University London Research Ethics Committee approved the experimental procedure and protocol. Practice approved under reference \textit{''SREC 15-16 01 E 29 09 201''} of the \nth{11} of November 2015. 

In this study, nine healthy volunteers participated. In total, six males and three females aged \numrange{23}{37} years old (mean 28.77) participated in the study. As per regulation of the committee, only healthy participants took part of the research. Any participant with cardiovascular disease history did not take in the study. 

Previous to the recruitment of participants, they received documentation explaining the whole procedure. Once they agreed, the party returned the consent form signed to schedule the study. The experiments took place at the Research Centre for Biomedical Engineering of City, University of London. Upon arrival partaker acclimatised for \SI{10}{\min} which room temperature was \SI{22}{\degreeCelsius}\SI{\pm 2}{\degreeCelsius}. During this period, it was clearly explained the experimental procedure to the attendant. Then the following steps took place.


%********************************** %First Section  **************************************
\section{Experimental procedure} %Section - 4.1
\label{section4.1}

After filling paperwork and completing acclimatisation different instruments were used to acquired physiological signals. These measurements included, ECG, PPG, laser Doppler, ultrasound Doppler and impedance plethysmography. 

\todo{Check if this is in the notation}

%********************************** % Third Section  *************************************
\subsection{Instruments setup}
\label{section4.1.1}

First, iPG data were from the left arm of all assistants. However, this measurement requires recording forearm's segment volume by weighing length and circumference. In fact, it can be calculated using equation \ref{eq:Volume}. Additional, distance from the heart to shoulder, upper arm length, and shoulder to index finger length was also recorded. 

\begin{align}
\label{eq:Volume}
V_e =\frac{l(C_1^2+C_1 C_2 + C_2^2)}{(12\pi)}
\end{align}

where $V_e$ is segment's volume, $l$ is the length between sensing electrodes, $C_1$ is the circumference at elbow's electrode and $C_2$ is the circumference at wrist's electrode.

Second, all participants sat in a comfortable chair. Their left arm was resting on a soft cushion on a table next to a chair adjusted to collaborator's height. Then, blood pressure was taken using an automated instrument recording diastolic and systolic values. Then, participants placed four ECG leads themselves, forming an Einthoven triangle. According to the device's instructions, electrodes must be placed one on each shoulder and ankle. Then, leads were secured to the electrodes verifying that ECG signal was clean. The apparatus includes an output port that exports the waveform for further processing.

\todo{Check process description for previous paragraph} 

\todo{reference of the blood pressure device}

Following this step, a PPG probe was placed on the index finger. A PPG instrument designed by the research group known as Zen PPG was used for this purpose. This equipment has two channels, but only one was required for the experiment procedure. It provides two analogue outputs containing AC and DC components of the photoplethysmography waveform. 

\todo{Check for the type of probe used. Nellcor?}

In the next step, the laser Doppler flowmetry probe was attached on the forearms' mid-section. This device measures small changes of blood flow over the vascular bed under the skin. The instrument counts with an analogue port that extracts the waveform for further processing.  

\todo{Check for maker of the Doppler Flowmetry device}

\todo{Check a small paragraph about how this device works and what is useful for}

Later, the ultrasound Doppler probe was placed close as possible to the radial artery. It produces an audible signal comparable with blood's speed at this point. For this, the device 's probe head requires being placed at a fixed angle. Using a pole the instrument's head was secured and placed on the party's wrist. The appliance also provides an external port for additional processing.  

\todo{add a paragraph description of how Ultrasound Doppler works}

\todo{adding the correct name of the pole}

Finally, impedance plethysmography electrodes were allocated. For this, ECG electrodes provided good contact area required for the experiment. Current probes were placed following the path of the left radial artery, one below the elbow (brachial artery) and a second on top of the radial artery close to the wrist. Sensing electrodes were located next to the previous electrodes towards the internal side of the forearm. The distance between sensing electrodes was recorded as the length of the volume segment. Also, arm's circumference at these electrodes position also was taken. Using equation xx is possible to estimate the volume of the segment.  The device as described in Section xx provides two channels, impedance baseline and plethysmography waveform. 

\todo{add reference to circumference equation}

\todo{add reference to the previous chapter describing the device}

%********************************** %Second Section  *************************************
\subsection{Data acquisition}
\label{section4.1.2}

As described previous section, all the instruments provided external ports for external data processing. These output ports were connected to a DAQ NI-6211 (National Instruments). This analogue to digital converter card provides 32* channels and a combined sampling rate of \SI{250}{\kilo\sample\per\second}. It connects to a personal computer via USB port (Version 1.0). All the signal were sampled at \SI{1}{\kilo\hertz}.

A virtual instrument using LabView was created for displaying, processing and storing raw data. Fig. Xx shows the front-end of the program. A custom virtual instrument was programmed showing four channels of waveforms. iPG, ECG, PPG and Ultrasound Doppler were used for displaying signals.

For display purposes, a band-pass filter was applied to the iPG and PPG signals. Nevertheless, the data recorded only was from the unfiltered waveforms. 

\todo{check what filters were applied to the signal}

Two tabs provided settings and voltage adjustment for PPG signal. The participant tab required the following data: left arm dimensions, blood pressure and path name. At the end of the experimental procedure, all raw data was stored on an LVM file created by LabView. 

\todo{Add proper description of the DAQ used, including channels}

\todo{Add reference of LabView}

\todo{Add image of the front end of the LabVIEW instrument}

%********************************** %Second Section  *************************************
\subsection{Experimental protocol}
\label{section4.1.3}

This part of the experiment aimed to test device?s performance for DC and AC signals. Hence measurements were taken from channel 1 using electrodes position reference E2 and E5 (Fig. 1). Once all electrodes were connected and secured 2 minutes of recordings were taken to test the performance of the setup. As soon as it was demonstrated that all signals were readable and clear 5 minutes of recordings were obtained followed by 3 minutes of venous occlusion at 20 mmHg below diastolic pressure recorded.

Pulse wave velocity (PWV) test was performed in just only one subject under same conditions explained previously. 




