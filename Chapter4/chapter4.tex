%!TEX root = ../thesis.tex
%*******************************************************************************
%*********************************** First Chapter *****************************
%*******************************************************************************

\chapter{Experimental Procedure}  %Title of the First Chapter

\ifpdf
    \graphicspath{{Chapter4/Figs/Raster/}{Chapter4/Figs/PDF/}{Chapter4/Figs/}}
\else
    \graphicspath{{Chapter4/Figs/Vector/}{Chapter4/Figs/}}
\fi


This section will describe the protocol and signal obtained during the device performance test. City University London Research Ethics Committee approved the experimental procedure and protocol. Practice approved under reference \textit{''SREC 15-16 01 E 29 09 201''} of the \nth{11} of November 2015. 

In this study, nine healthy volunteers participated. In total, six males and three females aged \numrange{23}{37} years old (mean 28.77) participated in the study. As per regulation of the committee, only healthy participants took part of the research. Any participant with cardiovascular disease history did not take in the study. 

Previous to the recruitment of participants, they received documentation explaining the whole procedure. Once they agreed, the party returned the consent form signed to schedule the study. The experiments took place at the Research Centre for Biomedical Engineering of City, University of London. Upon arrival partaker acclimatised for \SI{10}{\min} which room temperature was \SI{22}{\degreeCelsius}\SI{\pm 2}{\degreeCelsius}. During this period, it was clearly explained the experimental procedure to the attendant. Then the following steps took place.


%********************************** %First Section  **************************************
\section{Experimental procedure} %Section - 4.1
\label{section4.1}

After filling paperwork and completing acclimatisation different instruments were used to acquired physiological signals. These measurements included ECG, PPG, laser Doppler, ultrasound Doppler and impedance plethysmography. 


%**********************************% First subsection  *************************************
\subsection{Instruments setup}
\label{section4.1.1}

First, iPG data were from the left arm of all assistants. However, this measurement requires recording forearm's segment volume by weighing length and circumference. In fact, it can be calculated using equation \ref{eq:Volume}. Additional, distance from the heart to shoulder, upper arm length, and shoulder to index finger length was also recorded. 

\begin{align}
\label{eq:Volume}
V_e =\frac{l(C_1^2+C_1 C_2 + C_2^2)}{(12\pi)}
\end{align}

where $V_e$ is segment's volume, $l$ is the length between sensing electrodes, $C_1$ is the circumference of elbow's electrode and $C_2$ is the circumference of wrist's electrode.

Second, all participants sat in a comfortable chair. Their left arm was resting on a soft cushion on a table next to a chair adjusted to collaborator's height. Then, blood pressure was taken using an automated instrument recording diastolic and systolic values. Then, participants placed four ECG leads themselves, forming an Einthoven triangle. According to the device's instructions, electrodes must be placed one on each shoulder and ankle. Then, leads were secured to the electrodes verifying that ECG signal was clean. The apparatus includes an output port that exports the waveform for further processing.

\todo{Check process description for previous paragraph} 

\todo{reference of the blood pressure device}

Following this step, a PPG probe was placed on the index finger. The PPG instrument used is a design of the research group known as Zen PPG. This equipment has two channels, but the experiment only required one channel. It provides two analogue outputs containing AC and DC components of the photoplethysmography waveform. 

\todo{Check for the type of probe used. Nellcor?}
\todo{I'm not 100\% sure that has 2 outputs}

In the next step, the laser Doppler flowmetry probe was attached on the forearms' mid-section. This device measures small changes in blood flow over the vascular bed under the skin. The instrument counts with an analogue port that extracts the waveform for further processing.  

\todo{Check for maker of the Doppler Flowmetry device}

\todo{Check a small paragraph about how this device works and what is useful for}

Later, the ultrasound Doppler probe was placed close as possible to the radial artery. It produces an audible signal comparable with blood's speed at this point. For this, the device 's probe head requires being placed at a fixed angle. Using a pole the instrument's head was secured and placed on the party's wrist. The appliance also provides an external port for additional processing.  

\todo{add a paragraph description of how Ultrasound Doppler works}

\todo{adding the correct name of the pole}

Finally, impedance plethysmography electrodes were allocated. For this, ECG electrodes provided good contact area required for the experiment. Current probes were placed following the path of the left radial artery, one below the elbow (brachial artery) and a second on top of the radial artery close to the wrist. Sensing electrodes were located next to the previous electrodes towards the internal side of the forearm. The distance between sensing electrodes was recorded as the length of the volume segment. Also, arm's circumference at these electrodes position also was taken. Using equation xx is possible to estimate the volume of the segment.  The device as described in Section xx provides two channels, impedance baseline and plethysmography waveform. 

\todo{add reference to circumference equation}

\todo{add reference to the previous chapter describing the device}


%********************************** %Second subection *************************************
\subsection{Data acquisition}
\label{section4.1.2}

As outlined in the previous section, all the instruments provided external ports for external data processing. These output ports were connected to a DAQ NI-6211 (National Instruments). This analogue to digital converter card provides 32* channels and a combined sampling rate of \SI{250}{\kilo\sample\per\second}. It is connected to a personal computer via USB port (Version 1.0). All the signal were sampled at \SI{1}{\kilo\hertz}.

\todo{Maybe explain about the resolution of the DAQ. If this is a 16 bit the resolution is 2 to 16}

A virtual instrument using LabView was created for displaying, processing and storing raw data. Fig. Xx shows the front-end of the program. A custom virtual instrument was programmed showing four channels of waveforms. iPG, ECG, PPG and Ultrasound Doppler were used for displaying signals.

\todo{Add image of the front end of the LabVIEW instrument}

For display purposes, a band-pass filter was applied to the iPG and PPG signals. Nevertheless, the data recorded only was from the unfiltered waveforms because post-processing was required.  

\todo{check what filters were applied to the signal}

Two tabs provided settings and voltage adjustment for PPG signal. The participant tab required the following data: left arm dimensions, blood pressure and path name. At the end of the experimental procedure, all raw data was stored on an LVM file created by LabView. In the second tab, the VI adjusts the light intensity of the light PPG device. Usually a \SI{20}{\milli\ampere} is enough to provide a good quality signal. 

The main panel apart from just displaying waveforms also provides control buttons and a timer.  Pressing the start button starts a timer and waveforms begin to be stored in memory. Pushing on the same button again stops the timer and saves all the data in a local LVM file. 

\todo{Add proper description of the DAQ used, including channels}
\todo{Add reference of LabView in bibliography}

%********************************** %Third subection *************************************
\subsection{Experimental protocol}
\label{section4.1.3}

This experiment was aiming to look into waveform changes when an occlusion occurs. Achieving this is possible by applying a mechanical blockage using a standard blood pressure instrument. This device requires being pumped manually using an inflation bulb. The instrument also provides a gauge indicating cuff's pressure level. The cuff was secured to the left biceps of the participant. 

The protocol required recording three different types of blood flow occlusion. The first kind of blood flow restriction was venous occlusion.  This obstruction can be produced by inflating the cuff usually between \SIrange{10}{20}{\mmHg} below diastolic pressure. Thus, blocking the venous blood return from the forearm.  Overall, for each participator the target pressure was \SI{20}{\mmHg} under their diastolic pressure recorded at the beginning of the session. 

The second class of blood flow restriction was partial arterial occlusion. This kind of blockage decreases the amount of arterial blood coming into the forearm but also stops venous blood return. As an illustration, this can be obtained by applying a mechanical compression to the upper arm. The desired value should be between diastolic and systolic pressure. Therefore, the objective was to occlude in the mean value between diastolic and systolic pressure. For that, it was calculated using the following equation.

\todo{find a classy explanation about arterial blood occlusion, explaining that not allowing full compliance reduce flow}

\begin{align}
\label{eq:meanpressure}
P_m = \frac{P_d + P_s}{2}
\end{align}

where $P_d$ is diastolic pressure and $P_s$ is systolic pressure. 

The last kind of occlusion needed is total occlusion, which is possible to obtain by occluding above systolic pressure. In this experiment, blood flow was blocked inflating the cuff above \SI{20}{\mmHg} of participant's systolic value recorded. This kind occlusion completely restricts the inflow and outflow of venous and arterial blood. Hence, it is expected that not change of volume takes place.

\subsubsection{Recording waveforms}

At this point, as soon as all the instruments were operating successfully a small test took place. While waveforms were being shown on screen, \SI{2}{min} of recordings were taken but not recorded yet. Participants were reminded to limit their movements during the experiment. 

As soon as all signals were readable and free of noise, the start button of the VI was pushed to initiate the study. In the beginning, \SI{5}{\min} of baseline recordings were obtained. These waveforms provided the standard control wave to compare with occluded waveforms. Then, as soon as the timer reached this time the cuff was inflated rapidly at \SI{20}{\mmHg} below diastolic pressure. Therefore, producing venous occlusion and swelling the forearm. This level of blockage was held for \SI{3}{\min} followed by a swift cuff's deflation. 

Similarly, it was required repeating the procedure with partial arterial occlusion. As found in the literature, after an occlusion happens \SI{5}{\min} are needed to restore blood pressure. For this reason, cuff's air valve was left open bleeding all the air contained inside. After this period, the cuff was quickly inflated again until reaching the pressure calculate in equation \ref{eq:meanpressure}. The cuff's pressure was maintained at this level for \SI{3}{\min} proceeded by a quick pressure release. 

Last, for the purpose of total blood occlusion a similar method, was applied as before. One more time, \SI{5}{\min} of baseline waveforms were taken followed by a fleet cuff inflation to \SI{20}{mmHg} above diastolic pressure. This compression was kept for \SI{3}{\min}. At this point, maintaining this crushing effect can be painful after a couple of minutes. As a result, some of the helpers ended up re-accommodating and producing motion artefacts. 

Finally, the last \SI{5}{\min} of baseline waveforms were recorded to study the recovering effect. When time was up the stop button was pushed on saving all the data into an LVM file for further processing in Matlab. 

%********************************** % Swction section 4.2******************************************
\section{Data processing}
\label{section4.2}

Once data were stored in the local LVM file, it was post-processed in Matlab~\cite{MATLAB:2016}. In detail, a GUI Matlab program was created capable of comparing signals, apply filters, find peaks and windowed sections of data. A front-end of the application can be seen in Fig. xxx.

\todo{Create an image about the front-end of matlab}

First, in the beginning, it was required to import LVM file structure into Matlab. Due to the LVM file contained unwanted headers a reading text command could not parse the data into Matlab easily. Moreover, sampling at \SI{1}{\kilo\hertz} created files of approximately \SI{300}{MB}. Importing this file size terribly slowed down the workstation. For this reason, while importing into Matlab, data was decimated in a magnitude of ten. Consequently, each physiological waveform was reduced to less than \SI{1}{MB} of size. After all, having files of this extent improved processing speed and used fewer resources. 

\todo{create a table showing the structure of the LVM file}

This data decimation does not affect the waveform data.  In fact, according to Nyquist rate, data was being oversampled. Physiological signals are between \SIrange{1}{2}{\hertz} for signals depending on heart rate. According to Nyquist up to \SI{4}{\hertz} would have provided enough sampling for the waveforms. Nevertheless, a sampling of \SI{100}{\hertz} is sufficient to provide a high resolution. However, having oversampled signals was not a random case. Indeed it allows implementing high order digital filters with sharp cut-off frequency. 

\todo{Add a little bit of Nyquist theorem for oversampling}

As soon as decimated data was imported into Matlab, the parsing program converted each waveform into a Matlab formatted file (.mat). Furthermore, this waveform data can be imported faster into the GUI than raw files. In essence, filters, windowing and data processing can be performed on-line saving in computing resources. 

When importing data into the GUI program is needed to convert data into meaningful scales. All physiological data described in section \ref{section4.1.1} coming from the devices were in volts. Some data need conversions such as iPG, ultrasound and laser Doppler flowmetry. 

Impedance plethysmography signals needed to be converted from volts into current and impedance respectively. The iPG device provides a channel for current sense $(V_{IDC})$, which corresponds to the peak voltage value of the current being delivered by the instrument. This value can be calculated by knowing the gain of the different stages of the circuits. First, according to the circuit schematic current is being sensed by a \SI{10}{\ohm} resistor $(R_x)$. Then, this signal is amplified by an In-Amp which gain $(G_1)$ was 276. Following this stage, the signal is halved by a half-wave rectifier. Later, this wave is fully rectified using a peak detection circuit. In the final analysis, the current value can be calculated using equation \ref{eq:current}

\begin{align}
\label{eq:current}
I = \frac{V_{IDC} \times 2}{R_x \times G_1} = \frac{V_{IDC} \times 2}{276 \times\ 10 \Omega} = \frac{V_{IDC}}{1380}
\end{align}

\todo{add a reference toward the design of the device about the kind of data that ports provide. Current, AC and DC.}

\todo{Add a section in the design of the device showing how current waveform is converted from AC signal to DC and then into a value.}

\todo{add reference to the circuit that performs this task conversion task}

From the second channel $(V_{ZDC}$ of the iPG device the voltage coming from this port is equivalent to the impedance of the forearm. First, it is wanted to know the real voltage value  $(V_z)$ from the forearm segment read by the impedance device. Given this pint, as it can be seen from figure xxx, signal initially was amplified by In-Amp (IAxx). The gain of this stage was set to 35.65. Then the signal was rectified by the super-diode circuit. These two are the only components affecting the signal path. In conclusion, the voltage can be calculated by using equation \ref{eq:vzdc}.

\todo{add reference of a schematic showing the stages of the signal and a reference to the chip}

\begin{align}
\label{eq:vzdc}
V_z = \frac{V_{ZDC} \times 2}{35.65}
\end{align}

Finally, the impedance value can be computed by converting the voltage into impedance using ohm's law. This can be achieved by replacing equation  \ref{eq:current} and \ref{eq:vzdc} into \label{eq:impedance}.

\begin{align}
\label{eq:impedance}
Z = \frac{v}{i}=\frac{V_z}{I}
\end{align}

The third channel of the iPG device is the plethysmography waveform $V_{ZAC}$. This wave is an amplified version of the analogue signal contained within $V_{ZDC}$. Analysing signal's path, it can be noticed that two different circuits process the waveform. First, the signal comes from $V_{ZDC}$, meaning that the gain of IA-xxx also affects the signal $V_{ZAC}$. Following the circuit, the signal is filtered by a band-pass filter described in section 3.1.6. This filter has a gain in DC of 142.42. Then, the voltage of the plethysmography wave can be found by combining both gains. The calculation can be performed with equation \ref{eq:ViPG}.

\begin{align}
\label{eq:ViPG}
V_{iPG} = \frac{V_{ZAC} \times 2}{142.42 \times 35.65}=\frac{V_{ZAC}}{2538.66}
\end{align}

\todo{create a label and a link to the section 3.1.6}

Likewise, the impedance value of the dynamic signal can be calculated using ohm's law. By replacing the voltage value of the plethysmography signal from equation \ref{eq:ViPG} and the value of the current from equation \ref{eq:current}. 

\begin{align}
\label{eq:iPG}
Z_{iPG} = \frac{v}{i}=\frac{V_{iPG}}{I}
\end{align}

%********************************** % Swction section 4.2.1******************************************
\subsection{Digital filtering}
\label{section4.2.1}

The signals obtained from the devices contained variable sources of noise. For instance: respiration, mains noise, motion, high-frequency noise from the iPG current source (\SI{30}{\kilo\hertz}). To remove these sources of different noise filters were designed in Matlab for post-processing. The command \textit{designfilt} can create a variety type of filters. In the GUI any user can apply any filter on demand to any signal available. Table \ref{table:filters} overview the filters designed available for processing the waveforms. 

\begin{table}
\caption{Filters available from the GUI}
\centering
\label{table:filters}
\begin{tabular}{p{3.5cm} c c c c}
\toprule
\textbf{Filter}& \textbf{Order} & \textbf{Low cut frequency} & \textbf{High cut frequency} & \textbf{Other}\\
\midrule
Low-pass IIR & \nth{10} & -- & \SI{5}{\Hz} & --\\
\midrule
Band-pass IIR & \nth{10} & \SI{0.5}{\Hz} & \SI{5}{\Hz} & -- \\
\midrule
High-pass IIR & \nth{10} & \SI{0.5}{\Hz} & -- & --\\
\midrule
Band-stop IIR & \nth{10} & \SI{0.5}{\Hz} & \SI{5}{\Hz} & -- \\
\midrule
Savitzky-Golay & \nth{3} & -- & -- & Frame = 41\\
\midrule
Moving Average \newline (Simple) & -- & -- & -- & Lag = \SI{20}{\sec}\\
\bottomrule
\end{tabular}
\end{table}

According to the quality of the data, a specific kind of filter was applied to a distinct signal. For instance, raw iPG data contain high-frequency noise coming from the device main carrier frequency. That sort of noise is easily removable using the high-pass filter available. Moreover, combining filters is possible. For instance, the iPG waveform is extractable from the impedance signal. Applying a mix of band-stop and band-pass filters is possible to recover the signal $iPG_{AC}$.

Similar processing is viable with the PPG signals. From the DC signal is feasible to extract the AC waveform. Similarly, combining band-stop and band-pass filters.

In general, low-pass filters are useful to remove mains noise and high-frequency noises. Band-stop filters eliminate specific noise such as motion artefact and respiration but still DC is needed. A characteristic of this kind of noise is that is low frequency and usually below heart rate frequency. High-pass filters are practical to remove DC signals but still keeping high-frequency characteristics. 

%********************************** %Nomenclature found  *************************************
\nomenclature[z-ecg]{ECG}{Electrocardiography}
\nomenclature[z-ppg]{PPG}{Photoplethysmography}
\nomenclature[z-ipg]{iPG}{Impedance Plethysmography}
\nomenclature[z-daq]{DAQ}{Data Acquisition Card}
\nomenclature[z-usb]{USB}{Universal Serial Bus}
\nomenclature[z-vi]{VI}{Virtual Instrument}
\nomenclature[z-gui]{GUI}{Graphic User Interface}
\nomenclature[z-iir]{IIR}{Infinite Impulse Response}
