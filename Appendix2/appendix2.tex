%!TEX root = ../thesis.tex
% ******************************* Thesis Appendix B ********************************

\chapter{Arduino program to control DDS AD5930}
\label{Appendix: Arduino}

\lstset{language=Arduino} 
\begin{lstlisting}
/*
AD5930 Controller program

This program receives instructions via serial USB connection from LabView or Matlab.
It communicates vias SPI portocol to the AD5930.

The Host PC send commands such as frequency Lock, start/stop waveform.

modified 19 October 2015
by Mario Bejarano
City University London
*/

#include "SPI.h"
int ss=10;
int del=200;
int incomingByte = 0;
int CTRL_pin=8;
int IPin=4;
int VPin=3;
int PhasePin=2;
int Bank1 = 0;
int Bank2 = 1;
int val1=0;
int val2=0;
int val3=0;
int Battery1 = 0;
int Battery2 = 0;
int LED = 7;
String Z;
String frequ;

#define AD5930_F_MCLK 50000000          // 25MHz clock
#define AD5930_2POW28 16777216         // 2 to the power of 24

void setup()
{
	Serial.begin(115200);
	pinMode(VPin, INPUT);
	pinMode(IPin, INPUT);
	pinMode(PhasePin, INPUT);
	pinMode(Bank1, INPUT);
	pinMode(Bank2, INPUT);
	pinMode(ss, OUTPUT);
	pinMode(CTRL_pin, OUTPUT);
	pinMode(LED, OUTPUT);
	SPI.begin();
	SPI.setBitOrder(MSBFIRST); 
	SPI.setDataMode(SPI_MODE2);
	digitalWrite(ss, HIGH);
	digitalWrite(CTRL_pin, LOW);
}

void setValue(int value)
{
	digitalWrite(ss, LOW);
	SPI.transfer(0);
	SPI.transfer(value);
	digitalWrite(ss, HIGH);
}

void loop()
{
	delay(100);
	// send data only when you receive data:
	if (Serial.available() > 0) {
	// read the incoming byte:
	char command = Serial.read();           //Read command byte
	long incommingData = Serial.parseInt(); //Read data attached to it
	
	
	if (Serial.read() == '\n')			//Wait for newline (ENTER)
	{              
		//Serial.print(command);
		//Serial.print(incommingData);
		if (command == 'A')
		{
			//Serial.print("Data received:");
			//Serial.println(incommingData);
			calc_freq_reg(incommingData);
		}
		else if (command == 'B')
		{
			CTRL_signal();
		}
		else if (command == 'C')
		{
			val1 = analogRead(VPin);          
			Serial.println(val1);
		}
		else if (command == 'D') //Low BATTERY Control
		{
			val1 = analogRead(VPin);
			val2 = analogRead(PhasePin);
			String stringOne = "M";
			String stringTwo = stringOne + val1;
			//              Serial.println(stringTwo);
			String stringThree = "P";
			String stringFour = stringThree + val2;
			//              Serial.println(stringFour);
			String stringFive = stringTwo + stringFour;
			Serial.println(stringFive);
			if (val2 < 307) 
			{
				digitalWrite(LED,HIGH);
				Serial.println("Low Bat");
			}
			else
			{
				digitalWrite(LED,LOW);
				Serial.println("Good Bat");
			}
		}
		else if (command == 'E')
		{
			val1 = analogRead(VPin);
			val2 = analogRead(IPin);
			val3 = analogRead(PhasePin);
			String stringOne = "V";
			String stringTwo = stringOne + val1;
			//              Serial.println(stringTwo);
			String stringThree = "I";
			String stringFour = stringThree + val2;
			//              Serial.println(stringFour);
			String stringFive = "P";
			String stringSix = stringFive + val3;	
			String stringSeven = stringTwo + stringFour;
			String stringEight = stringSeven + stringSix;
			Serial.println(stringEight);
		}
		else if (command == 'F')
		{
			Battery1 = analogRead(Bank1);
			Battery2 = analogRead(Bank2);
			
			String stringOne = "B1:";
			float eq1 = (Battery1*5.0)/1024.0;
			Serial.println(eq1);
			float val1 = 3*eq1;
			String stringTwo = stringOne + val1;
			//              Serial.println(stringTwo);
			String stringThree = "B2:";
			float eq2 = (Battery2*5.0)/1024.0;
			Serial.println(eq2);
			
			float val2 = (eq2 - 3.70535)/0.18445;
			String stringFour = stringThree + val2;
			//              Serial.println(stringFour);
			String stringFive = stringTwo + stringFour;
			Serial.println(stringFive);
			if (abs(val1) < 12) 
			{
				digitalWrite(LED,HIGH);
				Serial.println("Low Bat Bank1");
			}
			else if (abs(val2) < 12) 
			{
				digitalWrite(LED,HIGH);
				Serial.println("Low Bat Bank2");
			}
			else
			{
				digitalWrite(LED,LOW);
				Serial.println("Good Bat");
			}
		}
		else if (command == 'X')
		{
			Serial.println("Ok");
		}
		else
		{
			Serial.println(" is an incorrect command ");
		}
		}
	}
}

void AD5930_Control(uint16_t control)
{
	control = control & 0x3FFF;         // Mask off upper 2 bits so we are writing to control register
	control = control | 0x2000;         // Ensure the B28 control register bit is set
	Serial.print("Control command :");
	Serial.println(control);
	digitalWrite(ss, LOW);              // Set FSYNC low
	delayMicroseconds(5);
	AD5930_Write(control);
	digitalWrite(ss, HIGH);             // Set FSYNC high
}

void AD5930_Write(uint16_t value)
{
	SPI.transfer(highByte(value));
	SPI.transfer(lowByte(value));
}

void AD5930_Phase(uint32_t phase)
{
	digitalWrite(ss, LOW);             // Set FSYNC low
	delayMicroseconds(5);
	AD5930_Write(0xC000 | (0x1FFF & (uint16_t)(phase)));
	AD5930_Write(0xE000 | (0x1FFF & (uint16_t)(phase>>12)));
	digitalWrite(ss, HIGH);            // Set FSYNC high
}

void AD5930_Raw(uint16_t value)
{
	digitalWrite(ss, LOW);             // Set FSYNC low
	delayMicroseconds(5);
	SPI.transfer(highByte(value));
	SPI.transfer(lowByte(value));
	digitalWrite(ss, HIGH);            // Set FSYNC high
}

void calc_freq_reg(long freq)
{
	int MSB;
	int LSB;
	
	AD5930_Raw(0x0FFF);                // Reset
	uint32_t AD5930Val = (uint32_t)(((freq/(double)AD5930_F_MCLK))*(double)AD5930_2POW28);
	MSB = (0xD000 | ((AD5930Val & 0xFFFF000)>>12));
	LSB = (0xC000 | (AD5930Val & 0x0FFF));
	AD5930_Raw(LSB);
	AD5930_Raw(MSB);	
	CTRL_signal();
}

void CTRL_signal()
{
	digitalWrite(CTRL_pin, HIGH);
	digitalWrite(CTRL_pin, LOW);
}

\end{lstlisting}

