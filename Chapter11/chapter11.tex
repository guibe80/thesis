%!TEX root = ../thesis.tex
%*******************************************************************************
%*********************************** Conclusions Chapter *****************************
%*******************************************************************************

\chapter{General conclusion and future work}  %Title of the First Chapter

\ifpdf
    \graphicspath{{Chapter11/Figs/Raster/}{Chapter11/Figs/PDF/}{Chapter11/Figs/}}
\else
    \graphicspath{{Chapter11/Figs/Vector/}{Chapter11/Figs/}}
\fi

%The motivation of this thesis of designing of a Bioelectrical impedance plethysmography capable of measuring changes in blood volume or flow has been accomplished. Furthermore, as the objectives implied, the designed device has the potential to be used in a home setting. First, evidentially the device operates with batteries, which it could be classified as a Class II device \cite{IEC60601}, because it is isolated from the and operated far from the mains. Thus, the instrument has the capability of being portable. Moreover, the design can be further miniaturised to improve its transportability. The experimental procedure demonstrated that the device detects both of the signals being acquired, the basal impedance and the dynamic signal of the arterial pulses. The quantification of the signals provided information of the changes of blood volume in percent units and estimation of blood flow in terms of litres per minute per volume of tissue. 

%The design of the instrument also included software that filtered unwanted noised in the signal and quantified the data in a swift manner. At present, it is only conceivable only off-line measurements. But it is possible that in the future work, with the algorithms implemented here it would be likely to measure beat by beat, changes of volume or quantifying blood flow instantaneously.


%****
%The secondary objectives of the thesis were focused on the analysis of the waveforms and its correlation to other instruments. As it was shown, all the devices used in the experiment provided data synchronised to the systolic peak. The following sections will give a further analysis of the findings from the report of these signals.
%*****


The main objective of this thesis was to present a home diagnostic system capable of measuring changes in blood volume - and by extension flow - using bioelectrical impedance. Such a system was realised in \ref{chapter design} in the form of a modular device that carried out the necessary measurements and exhibited the capability of being used in a home setting safely. The device worked as expected within the operational characteristics. Some key technological achievements and innovative features can be highlighted as follows: 

\begin{itemize}
	\item Some of the studies described in the literature were presented using impedance devices not explicitly designed for the measurement of changes of blood in limbs. For instance, some investigations used impedance cardiographers \cite{porter1985measurement, distefano1973bioelectrical, yamamoto1992impedance, couch1971noninvasive}, and other their circuits aimed just to measure either basal impedance \cite{mohapatra1979measurement, yamakoshi1980limb, nyober1950electrical, yamakoshi1978admittance, yamamoto1992impedance} or arterial pulses \cite{mohapatra1979measurement, costeloe1980continuous, corciova2011peripheral, brown1975impedance, wang2011development,yamamoto1992impedance}. The instrument presented in this thesis is focused solemnly on measuring impedance within the range of the tissue of the limbs \cite{gabriel1996dielectric}, measuring changes in the basal impedance and arterial pulses simultaneously. 
	\item Although some studies used their circuits or IC designs for measurement of bioelectrical impedance, most of their techniques were focused on quadrature demodulation and average DC value of the signal \cite{yufera2002integrated, pallas1993bioelectric, min2000lock}. The method used in this thesis uses the envelope demodulation with a super diode circuit which returns the peak value of the signals detected.
	\item Using envelope technique for measuring the current amplitude and voltage from the limbs reduces the need for calibration. The instrument measures continuously the amplitude of the current injected and the voltage sensed from the segment of the limb. Thus, the impedance was calculated directly from these two waveforms. 
	\item The iPG data presented in the literature are either basal impedance or arterial pulses. Some instruments extract the arterial pulses from the basal impedance which leads to low-quality waveforms \cite{mohapatra1979measurement, yamamoto1992impedance}. The instrument presented in this thesis separates the arterial pulses from the forearm impedance via hardware. Then, both signals are displayed separately offering the opportunity to analyse changes in both signals simultaneously.
\end{itemize} 

The innovation of the instrument presented in this thesis not only lies in the hardware design but also in the analysis of the impedance plethysmography waveform. The protocol of the experimental procedure described in section \ref{chapter procedure} affected the blood flow towards the limb. The instrument effectively detected changes in the basal impedance and the arterial pulses as shown in chapters \ref{chapter basal} and \ref{chapter apa} respectively. More important, clear changes were detected in most of the participants when performing venous and partial arterial occlusions. The author considers that according to the changes observed during the experimental procedure, the following innovations can be derived:

\begin{itemize}
	\item Although studies performing occlusive techniques have been carried out, these studies were focused mostly on the quantification of flow rate or the increase of volume during VOP \cite{mohapatra1979measurement, costeloe1980continuous, yamakoshi1980limb}. However, none of these studies has pointed out the rate of change during venous or partial arterial occlusion. The analysis presented in \ref{chapter basal} showed that the rate of change during partial arterial occurred swiftly than venous occlusion. This slope change may be one first indicator when an arterial blockage may be occurring when continuously monitored. 
	\item A study reported changes in the arterial pulses wave morphology during issues in the lymphatic system \cite{montgomery2011segmental}, but from the author's understanding, there was no evidence of analysis during a mechanical occlusion. As shown in this thesis, It is clear that the morphology of the waveform varies according to the type of occlusion applied to the upper arm. The data demonstrated that during both kinds of occlusions the systolic peak increased considerably. Thus, this might be an indicator or irregular blood flow towards the limb. Moreover, the data showed that there is a difference in amplitude at the dicrotic notch and the after pulse. During partial arterial occlusion, the amplitude of these points was considerably lower than venous occlusion. In fact, the information obtained demonstrated that during partial arterial occlusion the post dicrotic amplitude tended to be lower in magnitude than the dicrotic notch one. Therefore, this is a clear differentiator between both kinds of occlusion.
	\item To the author's knowledge, no other study had previously focused on methods for early detection of venous and arterial circulatory problems using iPG. After the quantitative and qualitative analysis of the basal impedance and the arterial pulses, it was put into consideration a ratio between both signals to identify faster alterations in the circulation. It was demonstrated that either basal impedance or APA morphology shifted during each blockage. By combining these two sets of information is possible to detect early changes in the blood circulation.
\end{itemize} 

\section{Future work}
The present work showed an initial prototype to measure iPG in an upper limb and the potential to detect circulatory problems in the limb. Nonetheless, some improvements in hardware and software would refine the full prototype. Furthermore, some experiments can be suggested to confirm if the body response causes changes in the iPG waveform which allows differentiating between arterial and venous components of the impedance signal. 

\subsection{Improvements of the iPG device}
The device had a good performance during all the tests performed. However, there are items of interest that could be implemented in future generations. From the design point of view, there is room to miniaturise the instrument. For this initial prototype, a modular approach was used. Nonetheless, designing an integrated PCB which could include the MCU on board as well as DAQ system will simplify the design considerably. Some challenges may arise for the design of the board. During the manufacturing of the PCB, it was found that combining digital with analogue channels increased the crosstalk between signals. Therefore, special care must be taken when designing a board of these characteristics. 

Complementary to this improvement, the device could accommodate a small set of batteries. This instrument included battery packs of \SI{12}{\volt} but if lower power banks were installed a better portability could be achieved. Furthermore, using a small current could be an option to use a smaller battery bank. However, additional challenges may arise. For instance, the impedance magnitude will get closer to the floor noise when using a  lower electrical current. Therefore, the noise will cover the signal of interest. 

Another improvement to the design will be the analysis of the phase of the signal. As most of the bioelectrical impedance devices, the instrument only measured the magnitude of the impedance. However, additional information about either volume or flow haemodynamics may lay under the phase shift data. Therefore, a channel dedicated to the phase could be an interesting add-on for future research. Nonetheless, there are quite a few challenges to achieve this. First, the phase shift at the frequency which the instrument was implemented is only \SI{12}{\degree} \cite{jaffrin1979quantitative}. Therefore, the level of noise at that level can be quite challenging. Possibly, using a high frequency could be an option, but then the real part of the signal will be affected.

\subsection{Waveform analysis under different conditions}
The use of additional instruments provided information on changes in blood volume, arterial flow and microcirculation. However, it could be of high interest to compare the iPG waveforms and NIRS method. The latter can provide data about tissue oxygenation, and within the basal impedance, reading yields information about the amount of blood in the tissue. Therefore, it can be established how effective is the iPG to measure perfusion changes compared to an optic method.

The analysis of the waveforms shown differences when calculating the blood flow during venous and partial arterial occlusion. The results presented all along the experiment suggested an increase of the blood flow which can only be explained by a vasodilation caused by a syncopal response. However, it could be recommended to perform cold tests to investigate if the narrowing of the vessels also displays changes in the dicrotic notch point.

It could be of high interest to analyse the waveform in the legs. Some illnesses like diabetic food require monitoring blood delivery towards the feet continuously. The instrument could be adjusted to take measurements within the impedance level needed to obtain a clear waveform signal. The changes in basal impedance and amplitude waveform may lead to an indication of poor perfusion in compromised limbs.


