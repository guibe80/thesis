%!TEX root = ../thesis.tex
%*******************************************************************************
%*********************************** Conclusions Chapter *****************************
%*******************************************************************************

\chapter{General conclusion and future work}  %Title of the First Chapter

\ifpdf
    \graphicspath{{Chapter11/Figs/Raster/}{Chapter11/Figs/PDF/}{Chapter11/Figs/}}
\else
    \graphicspath{{Chapter11/Figs/Vector/}{Chapter11/Figs/}}
\fi

The motivation of this thesis of designing of a Bioelectrical impedance plethysmography capable of gathering changes in blood volume or flow has been accomplished. The designed device has the potential to be used in a home setting. First, evidentially the device operates in batteries, which could be classified as a Class II device \mynote{Add reference for the technical norm.}. Thus, the instrument has the capability of being portable. Moreover, the design can be further miniaturised to improve its transportability. The experimental procedure demonstrated that the device detects both of the signals being enquired, the basal impedance and the dynamic signal of the arterial pulses. The quantification of the signals provided information of the changes of blood volume in percent units and estimation of blood flow in terms of litres per minute per volume of tissue. 

The design of the instrument also included software that helped to clean up the signal and fast processing the waveforms. At present, it is only possible to perform off-line measurements. But it is possible that in the future work, with the algorithms implemented here it would be possible to measure beat by beat, changes of volume or quantifying blood flow instantaneously.

The secondary objectives of the thesis were focused on the analysis of the waveforms and its correlation to other instruments. As it was shown, all the devices used in the experiment provided data synchronised to the systolic peak. The following sections will give a further analysis of the findings from the report of these signals. 

\section{Design of the bioelectrical impedance device}
This document described the design of the bioelectrical impedance device prototype as a proof of concept. A modular design was applied in the implementation of the prototype as the chapter \ref{chapter design} presented. The electrical characteristics of the bioelectrical impedance device included a wave generator capable of producing a sine waveform of up to \SI{50}{\kHz} which was designed using a DDS which can deliver even higher frequencies as described in section \ref{section DDS}. At this level of frequency, the current delivery should be bellow \SI{5}{\mA}. The device was designed to operate over four different levels of current between \SIlist{1.33;2.16;3.60;4.36}{\mA} which was achieved implementing a Modified Howland Circuit as presented in section \ref{section MHC}. The measurement of impedance was possible by measuring the current delivered and the potential measured. The sensing circuits were implemented with high input impedance instrumentation amplifiers assuring high noise rejection as described in chapter \ref{section V&I sense}. 

One of the exciting characteristics of the instrument is the expectation of producing not only the total impedance of the segment of the limb but also the arterial pulses amplitude. These two signals have been used separately one to measure changes in volume using venous occlusion plethysmography \cite{mohapatra1979measurement, costeloe1980continuous, yamakoshi1980limb} or estimation of blood flow beat by beat  \cite{corciova2011peripheral, porter1985measurement, brown1975impedance, marks1985computer}. However, the author considered that more information could be obtained when combined both measurements. Therefore, it was imperative to design a device that can provide both signals in real time, as the ones presented in the literature were just focused on one of the waveforms. 

Some principles were explored to check the viability of the instrument being used at home. Easy to use was one of the first challenges to be addressed. Bioelectrical impedance technology can be easily used without the assistance of specialised personnel. The sections \ref{section iPG electrodes} explained the benefit of using a tetrapolar electrodes configuration. In the implemented circuit this was the approach used as shown in \ref{section design electrodes}. Using four measurement points simplifies the intervention, the location of the electrodes could be a potential problem but having fixed accessible spots like elbows and wrists help always to take a similar measurement. In future implementations, this possible issue could be addressed by using a sleeve that covers the limb and marks the electrodes points position. 

Portability was another desired characteristic that may allow using the device at home. Using batteries to power the instrument helped to address not only this issue but also patient safety. As described in section \ref{section design battery} the device was powered using two battery banks of \SI{\pm 12}{\volt} which allowed the device to be charged and transporting it. Due that the device was operated as a battery-powered device, it allowed to use it safely in patients. The instrument was driving floating loads, meaning that current cannot run through earth creating a hazard to the patient. This characteristic was taken into account during the ethics approval, and all along the experimental procedure described in chapter \ref{chapter procedure}.

The design of the instrument cannot be completed without an adequate way to acquire data and to process the waveforms. The final prototype as described in section \ref{section design final} included a DAQ sampling the basal impedance and the APA signals at \SI{1}{\kHz}. However, as described in section \ref{section procedure 2} a lower frequency is enough to sample the signal. Therefore, in a future implementation, another method can be used saving physical and memory space. In the same chapter different techniques were presented to clean up the signals and quantifying the blood flow. Although these data were processed off-line, the algorithms are ready to be implemented in live data.

%********************************** %First Section  **************************************
\section{Detecting changes of volume using iPG} %Section - 10.1 
\label{section discussion 1}
As it has been presented in this document, there are two ways to calculate blood flow from the iPG data. For this, the segment of the body from where the measurement is taking place needs to represented as a cylinder. Luckily, this geometrical form is common in most parts of the body. For this study, the device was designed to measure the change of volume in the forearm. However, this could be calibrated to work in other parts of the body such as lower extremities eventually. 

The study performed in this document seeks to demonstrate how the iPG data could help to detect changes in volume due to venous and arterial occlusions. According to the literature, iPG can quantify changes of blood volume from the principle that the impedance of a body part is directly proportional to its volume as described by Nyober's equation  (see equation \ref{eq:nyober dV}). Hence, since the 1970's this theory has been tested assuring the effectiveness of the method \mynote{Add the reference found in the chapter 1}. 

There are different tissues in the body contributing to the impedimetric signal like fat, muscle, blood and bone. Most of these organs have intrinsic impedances when its movement is limited because as seen, the motion may affect impedance readings. The sum of all these resistive values is known as basal impedance. When an occlusion in the venous return occurs the total volume of the limb increases because of the pooling effect.

The instrument designed was able to detect the impedance from all the participants satisfactorily. In general, the basal impedance detected by the device was within the expected ranged found in the literature. However, the total impedance value becomes more important when examining inflammatory responses or analysis of ischemic tissue \mynote{I need to reference paper of lymphatic problems and ischemia} where changes of impedance are more drastic in time. 

From the point of view of detection of circulatory problems, the experimental task showed that is evident that impedance changes linearly when an occlusion in the forearm occurs. By occluding the venous return and not altering the arterial flow, blood can enter into the limb but cannot leave. As a result, there was a close linear increase in the volume of the arm. Hence, this gain of capacity can be quantified by the impedimetric method which is perceptible in the variation of the basal impedance. In fact, this is because the blood begins to cluster in the limb. As the blood's population increases, the conductivity of the forearm's section also rises in volume. Therefore, the resistivity proportionally decreases.

During the experiment presented in this document, the venous occlusion occurred throughout region 1 and region 2. As shown by figure \ref{fig:venous occlusion impedance}, it is clear that most of the participants experience a variation in their basal impedance readings. However, some of them were affected by motion which produced changes in the trend, like in participant 1. However, in most of the participants, there was a linear decrease in impedance. 

In a similar circumstance, there was a reduction in the forearm's inflow when the brachial artery in the upper arm was partially occluded at the midpoint of the participant's blood pressure. Performing this action stops the venous return but also alters the incoming arterial flow. As a result, again the forearm's volume increases shifting its resistivity. It was noticed that when this occurs, the basal impedance decreases in a larger slope. This greater slope might indicate a higher blood flow but this might be incorrect because restricting the brachial artery reduces the flow towards the distal section of the arm \cite{mccully2004muscle}. Thus, it seems that the increase of volume is related to a vasodilatory response. However. This effect can be confirmed by the data obtained from the Doppler ultrasound. In figure \ref{fig:DU flow} can be clearly seen that the participants experienced a decrease in their flow in the region 4 of the data between \SIrange{780}{960}{\second}.

\rvmynote{I need to check the information where the blood flow increased. The Volume increased but in this case cannot express as an increase in volume}

All along total occlusion (Region 6) seemed that there was not a clear change trend of the participant's basal impedance. Because both arterial and venous flow was blocked, there was no blood flow of any kind in the arm section. Therefore, this is the real basal impedance where all the tissues with their blood content were measured. The DC components of the PPG signal had a similar behaviour to this event where all participants had different responses. Indeed, only the Doppler ultrasound signal was able to reproduce a biological zero, the rest of the instruments showed some level of noisy data. However, when total occlusion occurs cells starvation is present and is expected ischemia to develop. Thus, it is expected that an increase of the impedance occurs in time \mynote{Find reference about impedance and ischemia}.

\section{iPG arterial pulses and the relation to other instruments measurements} %Section - 10.2
\label{section discussion 2}
The dynamic signal of the iPG waveform provided further beat by beat plethysmographic information. Correlating, the data from iPG with other instruments gave more details about contributors of the impedance signal. The two signals produced by bioelectrical impedance are related to a different physiological response like general tissue health and blood circulation given by the waveform of the arterial pulses. 

The instruments used in the experimental protocol looked into different physiological events occurring in the forearm. By locating the sensor head of the Ultrasound Doppler close to the radial artery provided information about the arterial circulation of the limb. The PPG sensor gave data of the changes in blood volume at the end of the extremity. Finally, the Laser Doppler flowmeter detected variations of blood flow at the micro-circulation level. 

This document presented the calculated blood flow from the iPG's arterial pulses using the method found in the literature \cite{corciova2011peripheral, porter1985measurement, brown1975impedance, marks1985computer}. According to Nyober's equation \ref{eq:nyober dV} the small change of impedance showed by the iPG's waveform can be converted into the blood flow per volume of tissue. The chapter \ref{section apa flow arterial pulses} described the results of turning the impedimetric systolic peak $\Delta R$ into the blood flow. This method quantifies the total blood including arterial and venous which feeds an specific volume of tissue in the body. Thus, the data showed by the instrument presented the contribution of both types of blood to the signal. 

By correlating the data obtained from the iPG arterial pulses and the other instruments is possible to establish the most significant contributors of the iPG signal. The data from the blood flow per volume of tissue quantified from the iPG data presented in chapter \ref{section apa flow arterial pulses} showed increments in their systolic peaks all along venous or partial arterial occurred. Indeed, the systolic pulses during partial arterial occlusion were slightly bigger than the ones during venous occlusion. This information contrasted with the data observed from the Doppler Ultrasound measurements. As described in chapter \ref{section comparison 2} the arterial flow is most affected when the blockage occurred above diastolic pressure. One could expect that the amplitude of the iPG's systolic pulses would also decrease during disturbances of the arterial flow as less blood enters into the limb, but it increased instead. One explanation for this could be that the body triggered a vasodilatory response when an occlusion occurred, where the cross-section area of the veins increments to accept more blood into the tissue. 

The response seen from the iPG signal is unique as the other instruments exhibited a decrease in their systolic peaks in similar proportion. For instance, as shown in \ref{section comparison LDF} and \ref{section comparison PPG} both signals displayed a reduction in their amplitudes during venous and partial arterial occlusion. This same response can be a result of the measurement of changes in the venous blood but not in the arterial. In fact, the decrease in the LDF amplitude suggests a reduction in the velocity of the RBC coming into the microcirculation. Additionally, the decline of the PPG amplitude indicates a shift in blood volume activated by the venous occlusion. In fact, during the experiment, the magnitude of PPG reduce significantly during partial arterial occlusion.

In general, it seems that the arterial blood is a significant contributor of the iPG's APA signal by not only detecting the arterial blood coming into the limb but is also giving an insight of how the tissue responded to a blood occlusion. The good correlation with the Ultrasound Doppler ($r^2 = 0.35$) showed in chapter \ref{section correlation 2}, illustrated that the iPG waveform amplitude changed with the arterial blood. Nevertheless, while reducing the arterial blood, the forearm's iPG waveform reacted differently causing an increment in the systolic magnitude which can be due to a venous response. On the other hand, when correlating the iPG data with PPG (see chapter \ref{section correlation 4}) and LDF (see chapter \ref{section correlation 3}) methods a low correlation was found ($r^2 < 0.08$) which apparently was caused by the different direction that the amplitudes took. While the iPG increased, the optical methods decreased. This reaction seems to indicate that the changes in venous blood presented as in increase in the iPG amplitude pulses, probably caused by an increase in diameter of the vessels during an occlusion. 

\section{Changes of the iPG waveform during venous and partial arterial occlusion}
The iPG signals provide further information about blood distribution during an occlusive events. The oscillations of the APA signal comes from the expansion of the arteries and venous during the heart cycle which creates tiny changes in the impedance. This small signal is contained within the basal impedance but is just a fraction of it. In fact, the contribution of this dynamic signal is just \SI{0.04}{\percent} to the total impedance. Hence, the device described in chapter \ref{chapter design} was able to isolate satisfactory this waveform and provide a high-resolution version of this signal.

Obtaining this level of detail was important as provided more features about changes in different parts of the impedance plethysmography waveform. The signal collected from this section of the arm gave characteristics of the circulatory process. Three reference points were identified as shown in \ref{section apa 1}. These markers included information about the systolic peak, the dicrotic notch and the diastolic peak which were distinguished by the algorithm implemented. Nonetheless, this waveform is unique to this set-up, changing the electrodes distance or using different frequencies may affect the impedance waveform.

It was observed that there were changes in the systolic peak (point A), dicrotic notch valley (point B) and diastolic peak (point C) when an occlusion occurs which also affected the blood flow response. During the process described in figure \ref{fig:iPG change points venous} where the changes between region 1, 2 and 3 took place, it can be seen that there is an apparent increase in the size of all the reference points in most participants during the occlusion followed by a return to baseline. 

However, this increase of impedance amplitude indicated a raise in the blood flow as reported by chapter \ref{section apa flow arterial pulses} but is not possible to establish whether is an increase in venous or arterial flows. The Doppler ultrasound referenced continuously to the radial artery blood flow. By performing the statistical study shown in \ref{section correlation 2} was established that there is some level of correlation between both signals ($r^2 = 0.35$). Nonetheless, during occlusions, there was a slight difference in the amplitude response. Comparing both flows rate (see figure \ref{fig:DU flow} and figure \ref{fig:blood_flow_plethysmography}) can be noticed that there was more sensitivity from the iPG than the DU at the time of the venous occlusion. From the Doppler ultrasound makes sense that the amplitude of the signal did not show marked changes during this occlusion since the arterial flow was not compromised. Although, the increase in the magnitude of the iPG waveform cannot be related to an increase in arterial flow but might be caused by a venous circulation response to the occlusion and picked up by the impedance device. 

On the other hand, during the partial arterial occlusion can be seen an alteration in the flow rate recorded by the Doppler ultrasound. The signal's magnitude from this device reduced at the time of this event. When analysing the partial arterial occlusion event (regions 3,4 and 5) described by section \ref{fig:iPG change points arterial} there was also an increase in points A  and B of the signal but point C reduced its amplitude. All these variations were common for most of the participants. The change of the diastolic peak is an indicator of an arterial problem from the waveform obtained. However, at the moment of calculating the blood flow from the impedimetric data a rise in the blood flow was reported which is not utterly concurrent with the DU's measurements. 

As a conclusion, the iPG device seems capable of measuring blood flow in healthy conditions with a good agreement with the amplitude of Doppler ultrasound measurements. However, there are variations in the basal impedance and the plethysmography waveform shape when a circulatory occlusion is present. In cases where arm's occlusion is not possible the best option is to calculate blood flow changes using the APA waveform. For this reason, monitoring the blood flow solely will not provide enough information to make a clinical decision additional data is required to identify where the real problem lies.

\rvmynote{Chack this last paragraph for a better understanding with the final conclusion}

%********************************** %Second Section  *************************************
\section{Basal impedance over plethysmography ratio} %Section - 7.2
\label{section discussion 4}

Despite the good level of correlation between DU's amplitude and the iPG's waveform, their blood flow rates do not provide enough details about venous or arterial occlusions. However, the information that the designed iPG device provides seems to present additional details that might give a clue about the kind of circulatory problem. 

As it was described before, the basal impedance has been used to find venous problems by using the venous occlusion technique. However, when an arterial blockage occurs there is a similar response that makes harder to identify where the problem prevails. On the other hand, the amplitude of the impedance plethysmography waveform might provide additional information to discriminate between both kinds of occlusions. Possibly, by combining these two data sets would be possible to discriminate between a right level of circulation and a venous and arterial circulatory problem. 

 \begin{figure}[!htpb]
 	\includegraphics[width=1\textwidth,keepaspectratio]{figure1}    
 	\caption[Ratio of the plethysmography waveform over basal impedance during the experiment]{Measurement of the ratio between the plethysmography waveform over basal impedance during the whole experiment. The blue line represents the systolic peaks, the yellow line the dicrotic notch and the red line the diastolic peak.}
 	\label{fig:ratio Z}
 \end{figure}

Therefore, it is proposed a ratio indicator between the plethysmography waveform and the basal impedance as a method to differentiate between venous and arterial problems. For this method to work, it is necessary to identify the three points of the plethysmographic waveform, systolic peak, dicrotic notch and diastolic peak. The calculation of the ratio can be performed using the following equation.

\begin{align}
	\label{eq:ratio Z}
	i_Z = \frac{Z_{PG}}{Z_{BAS}}
\end{align}


This is a dimensionless index with three indicators referencing each point of the plethysmography waveform. This method was applied to the participants in the study, and the results were portrayed in figure \ref{fig:ratio Z}. As the graph shows, for most of the data, the systolic peak is over the other two signals. When a venous or partial arterial occlusion occurs (regions 2 and 4), there is an increment in the index. 

The dicrotic notch and the diastolic peak seem to change as a matching pair for most of the baseline signals where the dicrotic notch looks like being slightly lower than the diastolic peak. However, at the time of partial arterial blockage, the systolic index drops below the dicrotic notch index. This event is a clear differentiator between both types of flow restriction.

%********************************** % Third Section  *************************************
\section{Option to evaluate blood obstructions using iPG DC and AC waveforms}  %Section - 7.3 
\label{section discussion 5}
As it was described from the estimation of the blood from the impedance signal could not be a good indicator of circulatory disturbances in either the arterial or venous path. One reason for this, it is that mathematical expression describes there is a direct relation between the amplitude of the wave and the estimation of blood flow. However, when a mechanical constriction of the arm occurs in the venous or arterial circulation the amplitude of the waveform increases. Therefore, this increase in the $\Delta R$ will be calculated as an increase in blood flood which is not altogether correct. 

For this reason, it is proposed a method where it is possible to detect problems in the circulatory path by quantifying and analysing differences in the waveform of the impedance plethysmography signal shape. As it was described in the section \ref{section apa 1}, there are three reference points that can be used to describe an impedance plethysmography waveform. Normally, a non disturbed waveform is represented as systolic peak (point A) higher than the diastolic peak (point C) and the dicrotic notch point (point B) lower than those peaks. 

However, as it has been noticed when a venous occlusion occurs, there are changes in the waveform's shape that are distinguishable with each phenomena. The first event that can be noticed is that the systolic peak increases in size. It seems that these gain of size is due to the blood pooling in the veins. Venous occlusion experienced this effect as well as the partial arterial occlusion. However, it must be noticed that partial arterial blockage is also a type of venous occlusion where it restricts the blood flow coming into the forearm. Hence, it is expected to also show an rise in the systolic peak. 

Also, during the venous occlusion can be seen that most of the participants experienced an increase in the magnitude of the dicrotic notch and the diastolic peak. Consequently, it can be said that increase in the impedance plethysmography waveform may represent restriction in the venous circulation towards the forearm. 

Nonetheless, this is not the only indicator of a venous circulatory problem. Likewise, the basal impedance also varies during this kind of occlusion because of the increase of blood volume in the veins. As a result, the basal impedance also decreases in time. Which is also an indicator of circulatory problem. So, presenting these two values in one quantifying number could provide a better indicator of blood flow restriction. 

\begin{figure}[!htpb]
	\includegraphics[width=1\textwidth,keepaspectratio]{figure2}    
	\caption[Bland and Altman plot of the relation between LDF and iPG]{Bland and Altman~\cite{bland1986statistical} plot of the relation between LDF and iPG. Data set corresponds to participants 2, 5, 6 and 7. The data has been normalised comparing the amplitude of both measurements. The different regions has been plotted with various colours and symbols to differentiate every event. The dotted line represents the perfect agreement, the dark line is the linear regression.}
	\label{fig:ration Z bar}
\end{figure}

\rvmynote{I have to double check the caption of this figure is not accurate}

\section{Future work}
The present work showed an initial prototype to measure iPG in an upper limb. There are improvements that can be done to the system from the hardware and software point of view. Furthermore, some experiments can be suggested to confirm if changes in the iPG waveform are caused by a body response or if there is a differentiator between arterial and venous components of the measurement. 

\subsection{Improvements of the iPG device}
The device had a good performance during all the tests performed. However, there are items of interest that could be implemented in future generations. From the design point of view, there is room to miniaturise the instrument. For this initial prototype, a modular approach was used. Nonetheless, designing an integrated PCB which could include the MCU on board as well as DAQ system will simplify the design considerably. Some challenges may arise for the design of the board. During the manufacturing of the PCB, it was found that combining digital with analogue channels increased the crosstalk between signals. Therefore, special care must be taken when designing a board of these characteristics. 

Complementary to this improvement, the device could accommodate a small set of batteries. This instrument included battery packs of \SI{12}{\volt} but if lower power banks were installed a better portability could be achieved. Furthermore, using a small current could be an option to use a smaller battery bank. However, additional challenges may arise. For instance, the impedance magnitude will get closer to the floor noise when using a  lower electrical current. Therefore, the noise will cover the signal of interest. 

Another improvement to the design will be the analysis of the phase of the signal. As most of the bioelectrical impedance devices, the instrument only measured the magnitude of the impedance. However, additional information about either volume or flow haemodynamics may lay under the phase shift data. Therefore, a channel dedicated to the phase could be an interesting add-on for future research. Nonetheless, there are quite a few challenges to achieve this. First, the contribution of the phase shift at the frequency used was about \SI{12}{\percent}. Therefore, the level of noise at that level can be quite challenging. Possibly, using a high frequency could be an option, but then the real part of the signal will be affected.

\subsection{Waveform analysis under different conditions}
The use of additional instruments provided information of changes in blood volume, arterial flow and micro-circulation. However, it could be of high interest to compare the iPG waveforms and NIRS method. The latter can provide data about tissue oxygenation, and within the basal impedance reading yields information about the amount of blood in the tissue. Therefore, it can be established how effective is the iPG to measure perfusion changes compared to an optic method.

The analysis of the waveforms showed differences when calculating the blood flow during venous and partial arterial occlusion. The results presented during the experimented suggests and increase of the blood flow which can only being explained by a vasodilation caused by a syncopal response. However, it could be recommended to perform cold tests to investigate if the narrowing of the vessels also display changes in the dicrotic notch point. 


